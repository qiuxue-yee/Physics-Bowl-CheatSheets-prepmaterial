\documentclass[12pt,a4paper]{article}
% Core packages
\usepackage[margin=1in]{geometry}
\usepackage{amsmath,amssymb,amsfonts}
\usepackage{lmodern}
\usepackage{graphicx}
\usepackage{enumitem}
\usepackage{xcolor}
\usepackage[most]{tcolorbox}
\usepackage[hidelinks]{hyperref}
\hypersetup{linktoc=all}
\usepackage{fancyhdr}
\usepackage{lastpage}
\usepackage{microtype}
\usepackage{needspace}
\usepackage{xparse}
\usepackage{xstring}
\usepackage{caption}
\usepackage{siunitx}
% Optional simple diagrams
\usepackage{tikz}
\usetikzlibrary{arrows.meta}

% Page style
\pagestyle{fancy}
\fancyhf{}
\fancyhead[L]{Ray Zhou}
\fancyhead[R]{Physics Bowl Cheat Sheets}
\fancyfoot[C]{Page \thepage\ of \pageref{LastPage}}
\renewcommand{\headrulewidth}{0.4pt}
\setlength{\headheight}{14.5pt}

% Plain style (used for TOC/title pages): hide header, keep custom footer
\fancypagestyle{plain}{%
  \fancyhf{}
  \fancyfoot[C]{Page \thepage\ of \pageref{LastPage}}
  \renewcommand{\headrulewidth}{0pt}
}

% Release mode (public vs internal)
\newif\ifpublicrelease
\publicreleasefalse % default: internal use

% Theme colors per Part (low-saturation, color-blind friendly)
\colorlet{ThemeFrame}{black!15}
\colorlet{ThemeTitle}{gray!15}
\newcommand{\setcheattheme}[2]{\colorlet{ThemeFrame}{#1}\colorlet{ThemeTitle}{#2}}
\newcommand{\parttheme}[1]{%% keys: mech|waves|em|optics
  \IfStrEqCase{#1}{%
    {mech}{\setcheattheme{teal!25}{teal!12}}%
    {waves}{\setcheattheme{yellow!25}{yellow!12}}%
    {em}{\setcheattheme{magenta!25}{magenta!12}}%
    {optics}{\setcheattheme{cyan!25}{cyan!12}}%
  }[\setcheattheme{black!15}{gray!15}]}

% Box styles
\tcbset{colback=white, boxrule=0.5pt}
\newtcolorbox{formulabox}{enhanced, breakable, colframe=ThemeFrame,
  colbacktitle=ThemeTitle, title=\textbf{Formulas \& Concepts},
  fonttitle=\bfseries, left=10pt, right=10pt, top=7pt, bottom=8pt,
  before skip=8pt, after skip=10pt}
\newtcolorbox{heuristicsbox}{enhanced, breakable, colframe=ThemeFrame,
  colbacktitle=ThemeTitle, title=\textbf{Heuristics \& Pitfalls},
  fonttitle=\bfseries, left=10pt, right=10pt, top=7pt, bottom=8pt,
  before skip=8pt, after skip=10pt}
\newtcolorbox{solutionbox}{enhanced, breakable, colback=white,
  colframe=black!20, colbacktitle=gray!12, title=\textbf{Solution},
  fonttitle=\bfseries, left=10pt, right=10pt, top=7pt, bottom=9pt,
  before skip=8pt, after skip=12pt}
\newtcolorbox{insightbox}{enhanced, breakable, colback=blue!2,
  colframe=blue!35, colbacktitle=blue!12, title=\textbf{Key Insight},
  fonttitle=\bfseries, left=8pt, right=8pt, top=5pt, bottom=6pt,
  before skip=6pt, after skip=8pt}

% Front-matter disclaimer style
\newtcolorbox{disclaimerbox}{enhanced, breakable, colback=gray!6,
  colframe=gray!45, left=10pt, right=10pt, top=7pt, bottom=7pt,
  before skip=6pt, after skip=8pt, fontupper=\small}

% Unit helpers
\newcounter{unitproblem}
\NewDocumentCommand{\Unit}{ O{} m }{%% [coverage-note] {title}
  \section{#2}%
  \setcounter{unitproblem}{0}%
  {\small\textit{Coverage checklist:} #1}\par\vspace{0.4em}
}
\NewDocumentEnvironment{cheatproblem}{ O{} }{% optional tag
  \refstepcounter{unitproblem}%
  \Needspace{12\baselineskip}% keep problem+solution close
  \par\noindent\textbf{Problem\ \#\theunitproblem.}\ %
}{\par}

% Knowledge point helpers (lightweight)
\makeatletter
\@ifundefined{KnowledgePoint}{%
  \NewDocumentEnvironment{KnowledgePoint}{ m }{\subsection{#1}}{}%
}{}
% To avoid duplicated headings with the tcolorboxes, keep these macros empty
\providecommand{\KPFormulas}{}
\providecommand{\KPHeuristics}{}
\providecommand{\KPProblems}{}
\makeatother

% Difficulty tags
\newcommand{\DOne}{\texorpdfstring{\textsuperscript{\textbf{D1}}}{ D1}}
\newcommand{\DTwo}{\texorpdfstring{\textsuperscript{\textbf{D2}}}{ D2}}

% Practice pointer macro (page is placeholder for now)
\newcommand{\practiceitem}[5]{%% Contest, Year, Division, Problem, Page/placeholder
  \IfStrEq{#5}{TBD}{\item #1\,\/\,#2\,\/\,#3\,\/\,#4}{\item #1\,\/\,#2\,\/\,#3\,\/\,#4\,\/\,Page: #5}}

% Title
\title{Physics Bowl D1 + D2 Cheat Sheets}
\author{Ray Zhou}
\date{Version 1.7}

\begin{document}
\begin{titlepage}
\thispagestyle{empty}
\centering
\vspace*{1.6cm}
{\Huge\bfseries Physics Bowl D1 + D2 Cheat Sheets\par}
\vspace{0.6cm}
{\Large Study Guide for Physics Bowl D1/D2\par}
\vspace{1.2cm}
{\Large Version 1.7\par}
\vfill
{\large Ray Zhou\par}
\vspace*{0.6cm}
\end{titlepage}
\ifpublicrelease
  \begin{center}
    \vspace{0.5em}
    {\small This work contains original problems and original explanatory text. Contest references in \textit{Practice Pointers} are metadata only (Contest/Year/Division/Problem/Page) and do not reproduce source items. Third-party contest materials remain the property of their respective owners.}
  \end{center}
\fi

% Use plain page style for front matter (title, letter, disclaimer, TOC)
\pagestyle{plain}
\pagenumbering{roman}

% --- Letter to the reader (before everything else) ---
\section*{A Letter to the Reader}
\addcontentsline{toc}{section}{A Letter to the Reader}
\begin{quote}
\raggedright
Dear Reader,\par\smallskip

Thank you for opening this guide. It aspires to be a quiet companion on your learning journey. Physics may seem like a towering mountain from afar, but up close, it resembles a garden.\par\medskip
May this book encourage you to pause and admire the flowers by the roadside, to look up at the starry sky and moonlight.

D1 is like each new encounter; D2 invites you to take one more step: perform a small derivation, use an approximation, or see the same landscape with greater clarity. Don't fear D2---view it as an invitation to pause, ask ``Why?'', and then lift your gaze to take in the scenery.\par\medskip
If you get stuck, try these three things: sketch another diagram; study a simpler boundary case first; change only one variable at a time. Most confusions will eventually clear.

May this guide reward your Curiosity, Creativity, and Caution. And may the world hold just a little more wonder. Wishing you smooth learning and steady progress.\par\medskip
\begin{flushright}
Ray Zhou
\end{flushright}
\end{quote}
\clearpage


% --- Reader guidance grouped under Reading Instructions ---
% --- Reader guidance placed before TOC ---
\section*{Reading Instructions}
\addcontentsline{toc}{section}{Reading Instructions}
\begin{itemize}[leftmargin=*]
  \item \textbf{Purpose}: This guidebook is a study guide to Physics Bowl D1/D2. Use it to go over key concepts and trends, then practice.
  \item \textbf{Structure}: Material is divided into parts (Mechanics, Dynamics, Waves/EM, Optics). Inside each part, boxes contain formulas, heuristics, and solutions.
  \item \textbf{Practice pointers}: The list items are metadata only (Contest/Year/Division/Problem/Page). Problem statements are in official sources.
  \item \textbf{Navigation}: There are hyperlinks provided throughout the PDF to facilitate faster jumps.
\end{itemize}


% --- Difficulty legend (D1/D2) ---
\subsection*{Difficulty Legend}
\addcontentsline{toc}{subsection}{Difficulty Legend}
\begin{itemize}[leftmargin=*]
  \item \textbf{\DOne}: Baseline knowledge and methods; no calculus derivations; aimed for D1 questions.
  \item \textbf{\DTwo}: More substantial topics and brief derivations (calculus/ODE/approximations); aimed for D2 or when principles are required.
  \item \textit{D1 chapters may list some approximation results for memorization purposes alone; derivations will be at \DTwo.}
\end{itemize}

% --- Notation \& Conventions (moved before TOC) ---
\subsection*{Notation \& Conventions}
\addcontentsline{toc}{subsection}{Notation and Conventions}
\begin{itemize}[leftmargin=*]
  \item \textbf{Units}: SI as default. Gravitation on Earth: $g=9.8\,\mathrm{m/s^2}$ unless otherwise stated. Always $\mathrm{m/s^2}$ (or $\mathrm{m\,s^{-2}}$). Angles: radians except when a symbol for degrees is shown.
  \item \textbf{Vectors}: Vectors are denoted by arrows (e.g., $\vec v$); magnitudes $v=\|\vec v\|$. Use components $v_x,v_y$ with $\hat\imath,\hat\jmath$.
  \item \textbf{emf vs field}: Use $\mathcal E$ for emf (scalar); $\vec E$ for electric field (vector). When magnitude is intended, write $|\vec E|$.\;\textit{Example (Faraday--Lenz):} $\mathcal E=\oint_{\partial S}\vec E\cdot d\vec l$ (emf) is to be distinguished from the field $\vec E$.
  \item \textbf{Common symbols}: Circuits use $V$ (voltage), $I$ (current), $R$ (resistance), $C$ (capacitance). Induced emf uses $\mathcal E$ (not $E$). Electric field is $\vec E$. Waves use $v$ (speed in medium), $f$ (frequency), $\lambda$ (wavelength).
  \item \textbf{Overloaded symbols}: $P$ for pressure (fluids) or power (circuits) depending on context. Thermodynamics: $\Delta U=Q+W$ (convention: $W>0$ means work done on the system).
  \item \textbf{Sign conventions}: Choose axes first and maintain consistency. For Doppler in a stationary medium, prefer the rule "approach increases frequency; recession decreases" and employ a sign diagram; see the Waves unit for the formula being given in detail. For KVL, direction of tracing loop; voltage rises are positive.
\end{itemize}

% --- Disclaimer (moved under Reading Instructions as a subsection) ---
\subsection*{Disclaimer}
\addcontentsline{toc}{subsection}{Disclaimer}
\begin{disclaimerbox}
This \textit{"Physics Bowl D1 + D2 Cheat Sheets"} (Version 1.7 by Ray Zhou) is an independent, non-official study guide created for educational purposes only. It summarizes key concepts, formulas, heuristics, and original solutions based on standard physics principles and publicly available Physics Bowl metadata (e.g., Contest/Year/Division/Problem/Page).\par\medskip
Full problem statements are sourced from official AAPT materials---please visit the American Association of Physics Teachers (AAPT) website at \href{https://www.aapt.org/programs/physicsbowl}{aapt.org/programs/physicsbowl} for complete exams and answers.\par\medskip
This guide is provided "as is" without warranties of accuracy or completeness. It is not endorsed by AAPT and should not be used as a substitute for official resources. Users should verify information independently. Distributed under CC BY-NC-SA 4.0 license for non-commercial, educational sharing.\par\medskip
For feedback or questions, contact \href{mailto:zhouxinrui2025@163.com}{zhouxinrui2025@163.com}, or visit the project on GitHub: \url{https://qiuxue-yee.github.io/Physics-Bowl-CheatSheets-prepmaterial/}.
\end{disclaimerbox}

% Ensure the Table of Contents starts on a fresh page
\clearpage

% --- Table of Contents (header suppressed during TOC pages) ---
\begingroup
  \pagestyle{plain}% suppress fancy header/footer on TOC pages
  \setcounter{tocdepth}{2}% up to subsections in TOC
  \tableofcontents
\endgroup
\clearpage
\pagenumbering{arabic}% main matter
\pagestyle{fancy}

% ===================== Part I: Mechanics =====================
\parttheme{mech}
\section*{Part I: Mechanics}
\addcontentsline{toc}{section}{Part I: Mechanics}

% -------- Unit 1: Kinematics --------
\Unit[Vectors vs scalars; constant-acceleration (SUVAT); projectile motion (same-level range/time/height); uniform circular motion ($a_c=v^2/r$, $T=2\pi r/v$); relative motion; calculus forms $v=dx/dt$, $a=dv/dt$ {\DTwo}; areas under $v$--$t$ and $a$--$t$ graphs {\DTwo}]{Unit 1: Kinematics}

\begin{KnowledgePoint}{Vectors \& Components \DOne}
  \KPFormulas
  \begin{formulabox}
  \textbf{Concept explanation:} Vectors have both magnitude and direction. The decomposition of a vector into orthogonal components translates geometric relations into algebraic relations such that calculations are simple component arithmetic.

  \textbf{Core formulas:}
  \[
  \left\{\begin{aligned}
    &\vec v=(v_x,v_y)\ (\text{2D}),\quad v=\|\vec v\|=\sqrt{v_x^2+v_y^2},\\
    &\text{Unit vectors: }\hat\imath=(1,0),\ \hat\jmath=(0,1),\ \vec v=v_x\hat\imath+v_y\hat\jmath.
  \end{aligned}\right.
  \]

  \textbf{Variable definitions:} $v_x,v_y$ scalar components; $v$ speed (magnitude of velocity); $\hat\imath,\hat\jmath$ orthonormal basis.


  \textbf{Prerequisites \& scope:} Axes must be orthogonal for Pythagorean magnitude; extend to 3D with $\hat k$ and $v=\sqrt{v_x^2+v_y^2+v_z^2}$.
  \end{formulabox}

  \KPHeuristics
  \begin{heuristicsbox}
  \begin{itemize}[leftmargin=*]
    \item Align axes with directions of steepest slopes and launch angles to reduce components.
    \item Separate vectors before writing equations (Newton's laws, kinematics) to prevent late trigonometry.
  \end{itemize}
  \vspace{0.4em}
  \begin{itemize}[leftmargin=*]
    \item Mixing magnitude and component equations. Solution: write separate equations for $x$ and $y$ and add if applicable.
  \end{itemize}
  \end{heuristicsbox}

  \KPProblems
\begin{cheatproblem}
  Resolve velocity vector $\vec v=(30\,\text{m/s},40\,\text{m/s})$ into magnitude and direction.
\begin{solutionbox}
  $v=\sqrt{30^2+40^2}=50\,\text{m/s}$. Direction $\theta=\arctan(40/30)\approx53.1^\circ$ above the horizontal.
\end{solutionbox}
\end{cheatproblem}
\begin{cheatproblem}
  A force of $100\,\text{N}$ acts at $30^\circ$ to the horizontal. Find its horizontal and vertical components.
\begin{solutionbox}
  $F_x=100\cos30^\circ=50\sqrt3\,\text{N}$, $F_y=100\sin30^\circ=50\,\text{N}$.
\end{solutionbox}
\end{cheatproblem}
\end{KnowledgePoint}

\begin{KnowledgePoint}{Constant Acceleration (SUVAT) \DOne}
  \KPFormulas
  \begin{formulabox}
  \textbf{Concept explanation:} Position and velocity are quadratic/linear in time with constant acceleration, providing closed-form relations (SUVAT equations).

  \textbf{Core formulas:}
  \[
  \left\{\begin{aligned}
    &x=x_0+v_0 t+\tfrac12 a t^2,\\
    &v=v_0+at,\\
    &v^2=v_0^2+2a\,(x-x_0).
  \end{aligned}\right.
  \]

  \textbf{Variable definitions:} $x$ position; $v$ velocity; $a$ constant acceleration; subscript $0$ initial value; $t$ time; $\Delta x=x-x_0$.


  \textbf{Prerequisites \& scope:} Acceleration must be constant over the interval; otherwise use calculus forms.
  \end{formulabox}

  \KPHeuristics
  \begin{heuristicsbox}
  \begin{itemize}[leftmargin=*]
    \item Choose the equation that excludes the unknown you lack (e.g., use $v^2$-form when time is absent).
    \item Work symbolically until the end to avoid compounding rounding errors.
  \end{itemize}
  \vspace{0.4em}
  \begin{itemize}[leftmargin=*]
    \item Applying constant-$a$ formulas when $a=a(t)$ or $a=a(v)$. Fix: switch to calculus forms or energy methods.
  \end{itemize}
  \end{heuristicsbox}

  \KPProblems
  \begin{cheatproblem}
  The velocity of a cart is given by the piecewise function $v(t)=\begin{cases} a t, & 0\le t\le T/2,\\ aT/2, & T/2< t\le T,\end{cases}$ with constant $a>0$. Find the total displacement in time $T$.
  \begin{solutionbox}
  Area under $v$--$t$: first triangle area $\tfrac12(a\,T/2)(T/2)=\tfrac{aT^2}{8}$; second rectangle area $(aT/2)(T/2)=\tfrac{aT^2}{4}$. Total $\Delta x=\tfrac{3aT^2}{8}$.
  \end{solutionbox}
  \end{cheatproblem}
\begin{cheatproblem}
  A car accelerates uniformly from rest to $20\,\text{m/s}$ in $4\,\text{s}$. Find the acceleration and distance traveled.
\begin{solutionbox}
  $a=\dfrac{v-v_0}{t}=\dfrac{20-0}{4}=5\,\text{m/s}^2$. Distance $\Delta x=v_0 t+\tfrac12 a t^2=0+\tfrac12\cdot5\cdot16=40\,\text{m}$.
  \end{solutionbox}
  \end{cheatproblem}
\end{KnowledgePoint}

\begin{KnowledgePoint}{Projectile (Equal Heights) \DOne}
  \KPFormulas
  \begin{formulabox}
  \textbf{Concept explanation:} Horizontal and vertical motions independent in uniform gravity. Decomposition into components provides closed-form solutions for flight time, range, and peak height at the same launch/landing height.

  \textbf{Core formulas:}
  \[
  \left\{\begin{aligned}
    &T=\frac{2v_0\sin\theta}{g},\\
    &R=\frac{v_0^2\sin2\theta}{g},\\
    &H=\frac{v_0^2\sin^2\theta}{2g}.
  \end{aligned}\right.
  \]

  \textbf{Variable definitions:} $v_0$ launch speed; $\theta$ launch angle; $g$ gravitational acceleration; $T$ flight time; $R$ range; $H$ apex height.

  \textbf{Prerequisites \& scope:} Launch/landing at equal heights; neglect air resistance; for unequal heights, solve quadratic in $t$.
  \end{formulabox}

  \KPHeuristics
  \begin{heuristicsbox}
  % (title removed per style unification)
  \begin{itemize}[leftmargin=*]
    \item Separate $x$ and $y$ equations; eliminate $t$ or use symmetry about the apex for time splits.
    \item For maximum range at equal heights, use $\theta=45^\circ$.
  \end{itemize}
  \vspace{0.4em}
  \begin{itemize}[leftmargin=*]
    \item Using $R$ formula when launch and landing heights differ. Fix: solve general quadratic and then compute $x(T)$.
  \end{itemize}
  \end{heuristicsbox}

  \KPProblems
  \begin{cheatproblem}
  A projectile is launched at speed $v_0$ and angle $\theta$ from flat ground and lands at the same height. Express the time of flight $T$, the maximum height $H$, and the range $R$ in terms of $v_0,\theta,g$.
  \begin{solutionbox}
  $T=\dfrac{2v_0\sin\theta}{g}$, $H=\dfrac{v_0^2\sin^2\theta}{2g}$, $R=\dfrac{v_0^2\sin2\theta}{g}$, by separating vertical and horizontal components and using constant-$g$ kinematics.
  \end{solutionbox}
  \end{cheatproblem}

  \begin{cheatproblem}
  Two projectiles launched with the same $v_0$ at angles $\theta$ and $90^\circ-\theta$ have equal ranges. What is the ratio of their maximum heights $H_\theta/H_{90^\circ-\theta}$?
  \begin{solutionbox}
  $H=\dfrac{v_0^2\sin^2\theta}{2g}$. Hence $\dfrac{H_\theta}{H_{90^\circ-\theta}}=\dfrac{\sin^2\theta}{\cos^2\theta}=\tan^2\theta$.
  \end{solutionbox}
  \end{cheatproblem}
\end{KnowledgePoint}

\begin{KnowledgePoint}{Uniform Circular Motion \DOne}
  \KPFormulas
  \begin{formulabox}
  \textbf{Concept description:} In uniform speed circular motion, the acceleration is centripetal (centerwards) with magnitude $v^2/r$; period and speed are related by circumference.

  \textbf{Core formulas:}
  \[
  \left\{\begin{aligned}
    &a_c=\frac{v^2}{r}=\frac{4\pi^2 r}{T^2},\\
    &T=\frac{2\pi r}{v}.
  \end{aligned}\right.
  \]

  \textbf{Variable definitions:} $r$ radius; $v$ speed; $T$ period; $a_c$ centripetal acceleration.

  \textbf{Prerequisites \& scope:} Speed constant; acceleration direction changes; for non-uniform circular motion add tangential component.
  \end{formulabox}

  \KPHeuristics
  \begin{heuristicsbox}
  % (title removed per style unification)
  \begin{itemize}[leftmargin=*]
    \item Draw radial/tangential components explicitly; set $a_t=0$ for UCM.
    \item Vertical circle normals: at the top, $T+mg=mv^2/r$; at the bottom, $T-mg=mv^2/r$.
    \item Minimum speed at the top for a taut string (no slack): $v_{\text{top}}\ge \sqrt{gr}$ (else $T=0$ at the top).
  \end{itemize}
  \vspace{0.4em}
  \begin{itemize}[leftmargin=*]
    \item Treating centripetal force as an extra force. Fix: centripetal is the net radial component of existing forces.
  \end{itemize}
  \end{heuristicsbox}

  \KPProblems
  \begin{cheatproblem}
  A bead moves in a circle of radius $r$ with period $T$. Compute its speed and centripetal acceleration.
  \begin{solutionbox}
  $v=\dfrac{2\pi r}{T}$ and $a_c=\dfrac{v^2}{r}=\dfrac{4\pi^2 r}{T^2}$.
  \end{solutionbox}
  \end{cheatproblem}
\end{KnowledgePoint}

\begin{KnowledgePoint}{Relative Motion \DOne}
  \KPFormulas
  \begin{formulabox}
  \textbf{Concept explanation:} Relative velocity cancels observer motion: the velocity of $A$ relative to $B$ is $\vec v_A-\vec v_B$.

  \textbf{Core formula:}
  \[
  \vec v_{A/B}=\vec v_A-\vec v_B.
  \]

  \textbf{Variable definitions:} $\vec v_{A/B}$ velocity of $A$ relative to $B$; $\vec v_A,\vec v_B$ velocities in an inertial frame.

  \textbf{Prerequisites \& scope:} Within Galilean (non-relativistic) regime; in 2D/3D apply component-wise.
  \end{formulabox}

  \KPHeuristics
  \begin{heuristicsbox}
  \begin{itemize}[leftmargin=*]
    \item Move to the target's rest frame to align directions and simplify timing.
    \item For winds/rivers, remove drift by aiming to oppose the current component.
    \item Boundary condition: if $v\le u$, landing directly across is not possible; you will be swept downstream and must land downstream.
  \end{itemize}
  \vspace{0.4em}
  \begin{itemize}[leftmargin=*]
    \item Superposing speeds scalarly when directions are different. Remedy: take difference of vectors component-wise.
  \end{itemize}
  \end{heuristicsbox}

  \KPProblems
  \begin{cheatproblem}
  A river of width $W$ flows east at speed $u$. A boat of speed $v$ relative to water aims at angle $\alpha$ north of west to land directly across. Find $\alpha$ and the crossing time, assuming $v>u$.
  \begin{solutionbox}
  Require zero east drift: west component equals current, so $v\cos\alpha=u$ and $\alpha=\arccos(u/v)$. North component is $v\sin\alpha=\sqrt{v^2-u^2}$, so time $t=\dfrac{W}{\sqrt{v^2-u^2}}$.
  \end{solutionbox}
  \end{cheatproblem}
\end{KnowledgePoint}

\begin{KnowledgePoint}{Calculus Forms \& Graph Areas \DTwo}
  \KPFormulas
  \begin{formulabox}
  \textbf{Concept explanation:} With acceleration varying, kinematics is found by integrating velocity and acceleration; graph areas are summaries of change.

  \textbf{Core formulas:}
  \[
  \left\{\begin{aligned}
    &v=\frac{dx}{dt},\quad a=\frac{dv}{dt}=\frac{d^2x}{dt^2}.\\
    &v(t)=v(t_0)+\int_{t_0}^t a(\tau)\,d\tau,\quad x(t)=x(t_0)+\int_{t_0}^t v(\tau)\,d\tau.\\
    &\Delta x=\int v\,dt\ (\text{area under }v\text{--}t),\quad \Delta v=\int a\,dt\ (\text{area under }a\text{--}t).\\
    &\text{Chain rule: } a=\frac{dv}{dt}=\frac{dv}{dx}\,v\ \Rightarrow\ v\,dv=a(x)\,dx.\\
    &\text{If }a=\text{const: } v^2=v_0^2+2a\,(x-x_0).
  \end{aligned}\right.
  \]

  \textbf{Variable definitions:} $x,v,a$ position/velocity/acceleration; $t$ time; integrals are definite over the time interval.

  \textbf{Prerequisites \& scope:} Differentiability over interval; interpret signed areas for direction-sensitive quantities.
  \end{formulabox}

  \KPHeuristics
  \begin{heuristicsbox}
  \begin{itemize}[leftmargin=*]
    \item Read slopes of $x$--$t$ as $v$ and slopes of $v$--$t$ as $a$; use areas for accumulated change.
    \item When $a=a(v)$ or $a=a(x)$, separate variables via $v\,dv=a\,dx$ to avoid time explicitly.
    \item If motion crosses turning points where $v=0$, integrate piecewise and track signs to avoid taking an incorrect branch.
  \end{itemize}
  \vspace{0.4em}
  \begin{itemize}[leftmargin=*]
    \item Confusing displacement with distance on $v$--$t$ when $v$ changes sign. Fix: integrate absolute value for distance.
  \end{itemize}
  \end{heuristicsbox}

  \KPProblems
  \begin{cheatproblem}
  An object has acceleration $a(t)=a_0+bt$ with constants $a_0,b$. If $x(0)=0$ and $v(0)=v_0$, find $x(t)$.
  \begin{solutionbox}
  Integrate: $v(t)=v_0+\int_0^t(a_0+b\tau)\,d\tau=v_0+a_0 t+\tfrac12 b t^2$. Then $x(t)=\int_0^t v(\tau)\,d\tau=v_0 t+\tfrac12 a_0 t^2+\tfrac16 b t^3$.
  \end{solutionbox}
  \end{cheatproblem}

  \begin{cheatproblem}
  A particle obeys $\dfrac{dx}{dt}=k\sqrt{x}$ with $k>0$ and $x(0)=0$. Find $x(t)$.
  \begin{solutionbox}
  Separate: $\dfrac{dx}{\sqrt{x}}=k\,dt\Rightarrow 2\sqrt{x}=kt+C$. With $x(0)=0$, $C=0$. Hence $x(t)=\dfrac{k^2 t^2}{4}$ and $v(t)=\dfrac{k^2 t}{2}$.
  \end{solutionbox}
  \end{cheatproblem}
\end{KnowledgePoint}

% (Removed outdated Practice Pointers for Part I Unit 1)

% -------- Placeholders for remaining Mechanics Units (skeletons only) --------
\Unit[Newton's laws; common forces (weight, spring, friction, normal); free-body diagrams; non-uniform circular motion; variable force and impulse]{Unit 2: Newtonian Dynamics}

\begin{KnowledgePoint}{Newton's Laws and Free-Body Diagrams \DOne}
  \KPFormulas
  \begin{formulabox}
  \[
  \left\{\begin{aligned}
    &\sum \vec F=m\vec a\ \text{; action--reaction pairs on separate bodies}.
  \end{aligned}\right.
  \]
  \end{formulabox}

  \KPHeuristics
  \begin{heuristicsbox}
  \begin{itemize}[leftmargin=*]
    \item Split each body; sketch clean FBDs and project along convenient directions.
    \item Don't place action--reaction pairs on the same drawing; they're on separate bodies and cancel only at the system level.
    \item Use radial/tangential axes for curves or inclined axes on ramps; use friction direction as unknown and solve for its sign.
    \item For connected bodies (strings/pulleys), apply kinematic constraints (e.g., equal string lengths imply proportional accelerations) and apply one tension per ideal massless string.
    \item In non-inertial frames (elevators, speeding cars), add $-m\,\vec a_{\text{frame}}$ only if you change frames explicitly.
  \end{itemize}
  \end{heuristicsbox}

  \KPProblems
\begin{cheatproblem}
  A $5\,\text{kg}$ block rests on a horizontal surface. A horizontal force $F=20\,\text{N}$ is applied. Find its acceleration if friction is negligible.
\begin{solutionbox}
  $\sum F=ma\Rightarrow 20=5a\Rightarrow a=4\,\text{m/s}^2$.
\end{solutionbox}
\end{cheatproblem}
\begin{cheatproblem}
  A $2\,\text{kg}$ block hangs from a string. Find the tension in the string when the block is at rest.
\begin{solutionbox}
  $\sum F_y=0\Rightarrow T-mg=0\Rightarrow T=2\cdot9.8=19.6\,\text{N}$.
\end{solutionbox}
\end{cheatproblem}
\end{KnowledgePoint}

\begin{KnowledgePoint}{Friction and Springs \DOne}
  \KPFormulas
  \begin{formulabox}
  \[
  \left\{\begin{aligned}
    &\text{Friction: } f_s\le \mu_s N\ \text{(variable up to max)},\quad f_k=\mu_k N,\\
    &\text{Spring force: } F=-kx\ \text{(Hooke) within elastic limit}.
  \end{aligned}\right.
  \]
  \end{formulabox}

  \KPHeuristics
\begin{heuristicsbox}
\begin{itemize}[leftmargin=*]
  \item Assume the direction of static friction to be unknown; solve and interpret its sign from the solution.
  \item Apply $f_s\le \mu_s N$; on the verge of sliding, take $f_{s,\max}=\mu_s N$. For steady sliding, use $f_k=\mu_k N$.
  \item For springs in series, use $k_{\text{eq}}$: in series $\tfrac{1}{k_{\text{eq}}}=\sum \tfrac{1}{k_i}$, in parallel $k_{\text{eq}}=\sum k_i$.
\end{itemize}
\end{heuristicsbox}

  \KPProblems
\begin{cheatproblem}
  A $3\,\text{kg}$ block on a horizontal surface has $\mu_k=0.2$. Find the kinetic friction force when the block slides.
\begin{solutionbox}
  $N=mg=3\cdot9.8=29.4\,\text{N}$. $f_k=\mu_k N=0.2\cdot29.4=5.88\,\text{N}$.
\end{solutionbox}
\end{cheatproblem}
\begin{cheatproblem}
  A spring with constant $k=200\,\text{N/m}$ is compressed by $x=0.1\,\text{m}$. Find the restoring force.
\begin{solutionbox}
  $F=-kx=-200\cdot0.1=-20\,\text{N}$ (opposite to compression).
\end{solutionbox}
\end{cheatproblem}
\end{KnowledgePoint}

\begin{KnowledgePoint}{Non-uniform Circular Motion \DOne}
  \KPFormulas
  \begin{formulabox}
  \[
  \left\{\begin{aligned}
    &\text{Radial: } \sum F_r=m v^2/r\ ;\ \text{tangential: } \sum F_t=m a_t.
  \end{aligned}\right.
  \]
  \end{formulabox}

  \KPHeuristics
  \begin{heuristicsbox}
  Split radial and tangential components explicitly; centripetal is net radial.
  \end{heuristicsbox}

  \KPProblems
\begin{cheatproblem}
  A car of mass $m$ rounds a curve of radius $r=50\,\text{m}$ at $v=20\,\text{m/s}$. Find the required centripetal force.
\begin{solutionbox}
  $F_c=\dfrac{mv^2}{r}=m\dfrac{400}{50}=8m\,\text{N}$.
\end{solutionbox}
\end{cheatproblem}
\begin{cheatproblem}
  A ball on a string swings in a vertical circle of radius $r$. At the top, the tension is $T$ and speed is $v$. Write the centripetal force equation.
\begin{solutionbox}
  At the top, $T+mg=\dfrac{mv^2}{r}$ (both point inward).
\end{solutionbox}
\end{cheatproblem}
\end{KnowledgePoint}

\begin{KnowledgePoint}{Impulse and Variable Forces \DTwo}
  \KPFormulas
\begin{formulabox}
  \[
  \left\{\begin{aligned}
    &\text{Impulse: } \vec J=\int \vec F\,dt=\Delta \vec p.
  \end{aligned}\right.
  \]
\end{formulabox}

  \KPHeuristics
  \begin{heuristicsbox}
  Extrapolate force-time profiles for varying forces; examine external impulses for system imparted momentum changes.
  \end{heuristicsbox}

  \KPProblems
\begin{cheatproblem}
  A constant force $F=10\,\text{N}$ acts on a mass for $\Delta t=3\,\text{s}$. Find the impulse delivered.
\begin{solutionbox}
  $J=F\Delta t=10\cdot3=30\,\text{N}\cdot\text{s}$.
\end{solutionbox}
\end{cheatproblem}
\begin{cheatproblem}
  A particle's momentum changes from $\vec p_i=(5,0)\,\text{kg·m/s}$ to $\vec p_f=(5,10)\,\text{kg·m/s}$ in $2\,\text{s}$. Find the average force.
\begin{solutionbox}
  $\vec J=\Delta\vec p=(0,10)\,\text{kg·m/s}$. $\vec F_{\text{avg}}=\dfrac{\vec J}{\Delta t}=(0,5)\,\text{N}$.
\end{solutionbox}
\end{cheatproblem}
\end{KnowledgePoint}

\Unit[Work $W=\int \vec F\cdot d\vec r$; Work--Energy theorem; potential energies (gravity, spring); mechanical energy conservation; power (instantaneous vs average); conservative fields ($F_x=-\,dU/dx$)]{Unit 3: Work, Energy, Power}

\begin{KnowledgePoint}{Work and the Work--Energy Theorem \DOne}
  \KPFormulas
\begin{formulabox}
  \[
  \left\{\begin{aligned}
    &\text{For constant }F\text{ and displacement }d\text{ at angle }\theta:\ \ W=Fd\cos\theta,\\
    &\text{Work--Energy: }\Delta K=W_{\text{net}},\\
    &\text{Power: instantaneous }P=\vec F\cdot\vec v,\quad \text{average over }\Delta t:\ \bar P=\dfrac{\Delta W}{\Delta t}.
  \end{aligned}\right.
  \]
\end{formulabox}

  \KPHeuristics
\begin{heuristicsbox}
\begin{itemize}[leftmargin=*]
  \item Choose the system to eliminate internal forces; only external work should be considered.
  \item Include nonconservative work explicitly (friction, thrust). For rolling without slipping, static friction often does no work on the rolling body.
  \item Check for indications by comparing the endpoints: $K_i+U_i$ vs. $K_f+U_f$.
\end{itemize}
\end{heuristicsbox}

  \KPProblems
\begin{cheatproblem}
  A $10\,\text{kg}$ block is pushed $5\,\text{m}$ horizontally by a constant force $F=30\,\text{N}$. Find the work done.
\begin{solutionbox}
  $W=Fd\cos\theta=30\cdot5\cdot\cos0=150\,\text{J}$.
\end{solutionbox}
\end{cheatproblem}
\begin{cheatproblem}
  A net force of $50\,\text{N}$ accelerates a $5\,\text{kg}$ mass from rest over $10\,\text{m}$. Find the final kinetic energy and speed.
\begin{solutionbox}
  $W_{\text{net}}=50\cdot10=500\,\text{J}$. By work-energy theorem, $\Delta K=W_{\text{net}}=500\,\text{J}$. Then $\tfrac12mv^2=500\Rightarrow v=\sqrt{200}=10\sqrt2\,\text{m/s}$.
\end{solutionbox}
\end{cheatproblem}
\end{KnowledgePoint}

\begin{KnowledgePoint}{Potential Energy and Conservation \DOne}
  \KPFormulas
\begin{formulabox}
  \[
  \left\{\begin{aligned}
    &U_g=mgh\ \text{(near-Earth)},\quad U_s=\tfrac12 kx^2,\\
    &\text{Conservation (when nonconservative work zero): }K_i+U_i=K_f+U_f.
  \end{aligned}\right.
  \]
  \end{formulabox}

  \KPHeuristics
  \begin{heuristicsbox}
  Use reference $U=0$ conveniently; only differences matter.
  \end{heuristicsbox}

  \KPProblems
\begin{cheatproblem}
  A $2\,\text{kg}$ block falls $h=5\,\text{m}$ from rest. Find its speed at the bottom using energy conservation.
\begin{solutionbox}
  $K_i+U_i=K_f+U_f\Rightarrow0+mgh=\tfrac12mv^2+0\Rightarrow v=\sqrt{2gh}=\sqrt{2\cdot9.8\cdot5}=\sqrt{98}\approx9.9\,\text{m/s}$.
\end{solutionbox}
\end{cheatproblem}
\begin{cheatproblem}
  A spring ($k=100\,\text{N/m}$) is compressed by $x=0.2\,\text{m}$. Compute the stored elastic potential energy.
\begin{solutionbox}
  $U_s=\tfrac12 kx^2=\tfrac12\cdot100\cdot0.04=2\,\text{J}$.
\end{solutionbox}
\end{cheatproblem}
\end{KnowledgePoint}

\begin{KnowledgePoint}{Conservative Fields \DTwo}
  \KPFormulas
\begin{formulabox}
  \[
  F_x=-\\,\\dfrac{dU}{dx}
  \]
  Path independence in conservative fields; use energy methods when applicable.
  \end{formulabox}

  \KPHeuristics
  \begin{heuristicsbox}
  Use energy methods if forces are conservative; verify curl-free areas.
  \end{heuristicsbox}

  \KPProblems
\begin{cheatproblem}
  A mass moves in a potential $U(x)=\tfrac12 kx^2$. Find the force acting on it.
\begin{solutionbox}
  $\vec F=-\dfrac{dU}{dx}\,\hat\imath=-kx\,\hat\imath$ (Hooke's law in 1D).
\end{solutionbox}
\end{cheatproblem}
\begin{cheatproblem}
  Show that gravitational potential $U=-\dfrac{GMm}{r}$ gives force $F=-\dfrac{GMm}{r^2}$ (radial).
\begin{solutionbox}
  First compute $\dfrac{dU}{dr}=\dfrac{d}{dr}\Big(-GMm\,r^{-1}\Big)=+\dfrac{GMm}{r^2}$. Therefore the radial force is $\vec F=-\dfrac{dU}{dr}\,\hat{\mathbf r}=-\dfrac{GMm}{r^2}\,\hat{\mathbf r}$ (attractive, inward).
\end{solutionbox}
\end{cheatproblem}
\end{KnowledgePoint}

\Unit[Momentum $\vec p=m\vec v$; impulse; momentum conservation; elastic/inelastic/fully inelastic collisions; center of mass; rocket equation]{Unit 4: Momentum \& Collisions}

\begin{KnowledgePoint}{Momentum and Impulse \DOne}
  \KPFormulas
  \begin{formulabox}
  \textbf{Concept explanation:} Momentum measures motion; impulse is the total effect of force over time and is equal to the change in momentum.

  \textbf{Core formulas:}
  \[
  \left\{\begin{aligned}
    &\vec p=m\,\vec v,\\
    &\vec J=\int_{t_1}^{t_2} \vec F\,dt=\Delta\vec p,\\
    &\text{System: }\ \vec P=\sum_i m_i\vec v_i,\quad \Delta\vec P=\vec J_{\text{ext}}.\ (\text{If }\vec J_{\text{ext}}=0,\ \vec P\ \text{conserved})
  \end{aligned}\right.
  \]

  \textbf{Variable definitions:} $\vec p$ momentum; $\vec P$ total momentum; $\vec J$ impulse; $\vec F$ external force; $m$ mass; $\vec v$ velocity.

  

  \textbf{Prerequisites \& scope:} Valid for Newtonian mechanics; for varying mass systems, take care with momentum flux (see rockets).
  \end{formulabox}

  \KPHeuristics
  \begin{heuristicsbox}
  \begin{itemize}[leftmargin=*]
    \item Identify a closed system (no external impulse) to use momentum conservation immediately.
    \item For short-duration large forces (collisions), use impulse–momentum rather than force–time details.
  \end{itemize}
  \vspace{0.4em}
  \begin{itemize}[leftmargin=*]
    \item Treating internal forces as external. Fix: define the system to include interacting bodies so internal forces cancel.
  \end{itemize}
  \end{heuristicsbox}

  \KPProblems
\begin{cheatproblem}
  A $4\,\text{kg}$ object moves with velocity $\vec v=(6,8)\,\text{m/s}$. Find its momentum vector and magnitude.
\begin{solutionbox}
  $\vec p=m\vec v=4(6,8)=(24,32)\,\text{kg·m/s}$, so $|\vec p|=\sqrt{24^2+32^2}=40\,\text{kg·m/s}$ and the direction matches $\vec v$.
\end{solutionbox}
\end{cheatproblem}
\begin{cheatproblem}
  A force $F(t)=10t\,\text{N}$ acts on a particle from $t=0$ to $t=2\,\text{s}$. Find the impulse delivered.
\begin{solutionbox}
  $J=\int_0^2 10t\,dt=\left[5t^2\right]_0^2=20\,\text{N}\cdot\text{s}$.
\end{solutionbox}
\end{cheatproblem}
\end{KnowledgePoint}

\begin{KnowledgePoint}{Elastic and Inelastic Collisions \DOne}
  \KPFormulas
  \begin{formulabox}
  \textbf{Concept explanation:} Collisions conserve total momentum; elastic ones also conserve kinetic energy. The center-of-mass (COM) frame makes algebra simpler.

  \textbf{Core formulas (1D):}
  \[
  \left\{\begin{aligned}
    &\text{Momentum: } m_1u_1+m_2u_2=m_1v_1+m_2v_2,\\
    &\text{Elastic energy: } \tfrac12 m_1u_1^2+\tfrac12 m_2u_2^2=\tfrac12 m_1v_1^2+\tfrac12 m_2v_2^2,\\
    &\text{Result (elastic): } v_1=\frac{(m_1-m_2)u_1+2m_2u_2}{m_1+m_2},\ \ v_2=\frac{2m_1u_1+(m_2-m_1)u_2}{m_1+m_2},\\
    &\text{Completely inelastic: stick }(v_1=v_2=v)=\frac{m_1u_1+m_2u_2}{m_1+m_2}.\\
  \end{aligned}\right.
  \]

  \textbf{Variable definitions:} $u_i$ initial, $v_i$ final velocities; $m_i$ masses; all along one line (1D).

  

  \textbf{Prerequisites \& scope:} For oblique/2D, conserve vector momentum and use geometry; kinetic energy changes via deformation/heat in inelastic cases.
  \end{formulabox}

  \KPHeuristics
  \begin{heuristicsbox}
  \begin{itemize}[leftmargin=*]
    \item Switch to the COM frame where total momentum is zero; velocities reverse in elastic 1D collisions.
    \item For 2D glancing collisions, conserve components along orthogonal axes; use restitution or geometry for angles.
  \end{itemize}
  \vspace{0.4em}
  \begin{itemize}[leftmargin=*]
    \item Enforcing kinetic energy conservation in inelastic impact. Fix: only momentum is guaranteed; account for energy loss.
  \end{itemize}
  \end{heuristicsbox}

  \KPProblems
\begin{cheatproblem}
  Two carts collide elastically in 1D. Mass $m_1=2\,\text{kg}$ at $u_1=5\,\text{m/s}$ hits mass $m_2=3\,\text{kg}$ at rest. Find $v_1$ and $v_2$.
\begin{solutionbox}
  $v_1=\dfrac{(2-3)\cdot5+2\cdot3\cdot0}{5}=\dfrac{-5}{5}=-1\,\text{m/s}$. $v_2=\dfrac{2\cdot2\cdot5+(3-2)\cdot0}{5}=\dfrac{20}{5}=4\,\text{m/s}$.
\end{solutionbox}
\end{cheatproblem}
\begin{cheatproblem}
  In a perfectly inelastic collision, $m_1=4\,\text{kg}$ at $u_1=6\,\text{m/s}$ collides with $m_2=2\,\text{kg}$ at rest. Find the final velocity.
\begin{solutionbox}
  $v=\dfrac{m_1u_1+m_2u_2}{m_1+m_2}=\dfrac{4\cdot6+0}{6}=4\,\text{m/s}$.
\end{solutionbox}
\end{cheatproblem}
\end{KnowledgePoint}

\begin{KnowledgePoint}{Center of Mass and System Dynamics \DOne}
  \KPFormulas
  \begin{formulabox}
  \textbf{Conceptual explanation:} The center of mass (COM) is an average of the mass distribution of a system; external forces accelerate the COM as if all the mass were centered at the COM.

  \textbf{Core formulas:}
  \[
  \left\{\begin{aligned}
    &\vec R=\frac{\sum m_i\vec r_i}{\sum m_i},\quad \vec V=\dot{\vec R}=\frac{\sum m_i\vec v_i}{M},\\
    &M\,\ddot{\vec R}=\sum \vec F_{\text{ext}}.\ (\text{Internal forces cancel in pairs})
  \end{aligned}\right.
  \]

  \textbf{Variable definitions:} $\vec r_i,\vec v_i$ positions/velocities; $M=\sum m_i$ total mass; $\vec R$ COM position.

  

  \textbf{Prerequisites \& scope:} Requires Newton's third law in internal pairs; for variable mass, include momentum flux.
  \end{formulabox}

  \KPHeuristics
  \begin{heuristicsbox}
  \begin{itemize}[leftmargin=*]
    \item Compute COM motion to track overall translation, then analyze internal relative motion separately.
    \item In explosions/fragmentation, the COM continues with pre-event velocity if external forces are negligible.
  \end{itemize}
  \vspace{0.4em}
  \begin{itemize}[leftmargin=*]
    \item Treating internal impulses as changing COM momentum. Fix: only external impulse changes total momentum.
  \end{itemize}
  \end{heuristicsbox}

  \KPProblems
\begin{cheatproblem}
  Two masses $m_1=3\,\text{kg}$ at $x_1=0$ and $m_2=2\,\text{kg}$ at $x_2=5\,\text{m}$ lie on a line. Find the COM position.
\begin{solutionbox}
  $X=\dfrac{m_1x_1+m_2x_2}{m_1+m_2}=\dfrac{0+10}{5}=2\,\text{m}$.
\end{solutionbox}
\end{cheatproblem}
\begin{cheatproblem}
  A system has two equal masses $m$ moving at velocities $\vec v_1=(2,0)$ and $\vec v_2=(-1,3)$ m/s. Find the velocity of the COM.
\begin{solutionbox}
  $\vec V=\dfrac{m\vec v_1+m\vec v_2}{2m}=\dfrac{(1,3)}{2}=(0.5,1.5)\,\text{m/s}$.
\end{solutionbox}
\end{cheatproblem}
\end{KnowledgePoint}

\begin{KnowledgePoint}{Variable Mass and Rockets \DTwo}
  \KPFormulas
\begin{formulabox}
  \textbf{Concept explanation:} For mass-exchange systems (rockets), momentum conservation for rocket+exhaust yields logarithmic change in velocity.
  
  \textbf{Core formulas:}
  \[
  \left\{\begin{aligned}
    &\text{Continuous: } m\,d\vec v=-\,\vec v_e\,dm,\\
    &\text{with } dm<0\ (\text{mass loss}),\ \vec v_e\ \text{ exhaust speed relative to rocket},\\
    &\text{Ideal Tsiolkovsky (1D): } \Delta v=v_e\ln\frac{m_i}{m_f}.\\
  \end{aligned}\right.
  \]
  
  \textbf{Variable definitions:} $\vec v_e$ exhaust velocity relative to rocket (magnitude $v_e$); $m_i,m_f$ initial/final mass; $m$ instantaneous mass.
  
  
  
  \textbf{Prerequisites \& scope:} Neglect external forces (or include gravity drag separately); $v_e$ constant; exhaust ejected at steady relative speed.
\end{formulabox}

  \KPHeuristics
  \begin{heuristicsbox}
  \begin{itemize}[leftmargin=*]
    \item Include gravity losses by subtracting $g\,\Delta t$ from $\Delta v$ when appropriate (vertical ascent approximation).
    \item Use staging by summing $v_e\ln(m_i/m_f)$ per stage.
  \end{itemize}
  \vspace{0.4em}
  \begin{itemize}[leftmargin=*]
    \item Using exhaust speed relative to Earth instead of rocket. Fix: $v_e$ is defined relative to the rocket.
  \end{itemize}
  \end{heuristicsbox}

  \KPProblems
\begin{cheatproblem}
  A rocket has initial mass $m_i=1000\,\text{kg}$, final mass $m_f=400\,\text{kg}$, and exhaust speed $v_e=2000\,\text{m/s}$. Find $\Delta v$ in space (ignoring gravity).
\begin{solutionbox}
  $\Delta v=v_e\ln\dfrac{m_i}{m_f}=2000\ln\dfrac{1000}{400}=2000\ln2.5\approx1833\,\text{m/s}$.
\end{solutionbox}
\end{cheatproblem}
\begin{cheatproblem}
  A rocket ejects mass at rate $\dot m=10\,\text{kg/s}$ with effective exhaust speed $v_e=1500\,\text{m/s}$. Find the instantaneous thrust force.
\begin{solutionbox}
  Thrust $=\dot m\,v_e=10\cdot1500=15000\,\text{N}$.
\end{solutionbox}
\end{cheatproblem}
\end{KnowledgePoint}

\Unit[Angular kinematics; torque ($|\boldsymbol{\tau}|=rF\sin\theta$); moment of inertia; $\tau_{\text{net}}=I\alpha$; rotational kinetic energy; angular momentum and conservation; rolling without slipping; inertia integrals]{Unit 5: Rotational Motion}

\begin{KnowledgePoint}{Angular Kinematics and Dynamics \DOne}
  \KPFormulas
  \begin{formulabox}
  \textbf{Concept description:} Rotational motion is analog to linear motion: torque is the rotational analog to force, and moment of inertia of mass.

  \textbf{Core formulas:}
  \[
  \left\{\begin{aligned}
    &\omega=\frac{d\theta}{dt},\quad \alpha=\frac{d\omega}{dt},\ \text{with constant-}\alpha\ \text{kinematics}.\\
&\boldsymbol{\tau}=\vec r\times\vec F,\quad |\boldsymbol{\tau}|=r_\perp F=rF\sin\theta,\quad \sum \tau=I\alpha,\\
    &K_r=\tfrac12 I\omega^2,\quad P=\tau\,\omega.
  \end{aligned}\right.
  \]

  \textbf{Variable definitions:} $\theta,\omega,\alpha$ angular position/velocity/acceleration; $\tau$ torque; $I$ moment of inertia.

  
  
  \textbf{Prerequisites \& scope:} Rigid body about a fixed axis; $I$ constant in time.
  \end{formulabox}

  \KPHeuristics
  \begin{heuristicsbox}
  \begin{itemize}[leftmargin=*]
    \item Use the perpendicular lever arm $r_\perp$ for torques; sum about convenient pivots to kill unknown forces.
    \item Prefer energy when forces are complicated but conservative; otherwise use $\sum \tau=I\alpha$ about the COM or a fixed axis.
  \end{itemize}
  \vspace{0.4em}
  \begin{itemize}[leftmargin=*]
    \item Mixing signs of torques from different reference senses. Fix: choose a positive rotation sense and stick with it.
  \end{itemize}
  \end{heuristicsbox}

  \KPProblems
\begin{cheatproblem}
  A disk of radius $r=0.5\,\text{m}$ and moment of inertia $I=2\,\text{kg·m}^2$ experiences a net torque $\tau=10\,\text{N·m}$. Find its angular acceleration.
\begin{solutionbox}
  $\alpha=\dfrac{\tau}{I}=\dfrac{10}{2}=5\,\text{rad/s}^2$.
\end{solutionbox}
\end{cheatproblem}
\begin{cheatproblem}
  A flywheel rotating at $\omega=20\,\text{rad/s}$ has $I=5\,\text{kg·m}^2$. Find its rotational kinetic energy.
\begin{solutionbox}
  $K_r=\tfrac12 I\omega^2=\tfrac12\cdot5\cdot400=1000\,\text{J}$.
\end{solutionbox}
\end{cheatproblem}
\end{KnowledgePoint}

\begin{KnowledgePoint}{Angular Momentum and Conservation \DOne}
  \KPFormulas
  \begin{formulabox}
  \textbf{Explanation of concept:} Angular momentum is conserved when no external torque is applied.

  \textbf{Core formulas:}
  \[
  \left\{\begin{aligned}
    &\vec L=I\,\vec\omega\ (\text{about fixed axis}),\ \ \sum \tau_{\text{ext}}=\frac{d\vec L}{dt}.\\
    &\text{If }\sum \tau_{\text{ext}}=0,\ \vec L\ \text{conserved}.
  \end{aligned}\right.
  \]

  \textbf{Variable definitions:} $\vec L$ angular momentum; $I$ moment of inertia.

  \textbf{Prerequisites \& scope:} Axis and point of reference must be specified; rolling applies at instantaneous point of contact.
  \end{formulabox}

  \KPHeuristics
  \begin{heuristicsbox}
  \begin{itemize}[leftmargin=*]
    \item For isolated systems with negligible external torques, apply $\vec L$ conservation about a fixed axis.
    \item Choose the reference point wisely to eliminate unknown torques.
  \end{itemize}
  \vspace{0.4em}
  \begin{itemize}[leftmargin=*]
    \item Forgetting that static friction can act either way in rolling. Fix: determine its direction from torque/acceleration requirements.
  \end{itemize}
  \end{heuristicsbox}

  \KPProblems
\begin{cheatproblem}
  A disk of radius $R$ rolls without slipping at $v_{cm}=5\,\text{m/s}$. Find $\omega$.
\begin{solutionbox}
  $v_{cm}=\omega R\Rightarrow\omega=\dfrac{v_{cm}}{R}=\dfrac{5}{R}\,\text{rad/s}$.
\end{solutionbox}
\end{cheatproblem}
\begin{cheatproblem}
  A solid sphere ($I=\tfrac25 mR^2$) rolls at $\omega=10\,\text{rad/s}$ and $v_{cm}=2\,\text{m/s}$. Find its total kinetic energy if $m=3\,\text{kg}$.
\begin{solutionbox}
  Using no-slip $v_{cm}=\omega R$, we have $I\omega^2=\tfrac25 mR^2\omega^2=\tfrac25 m v_{cm}^2$. Thus
  $K=\tfrac12 m v_{cm}^2+\tfrac12 I\omega^2=\Big(\tfrac12+\tfrac15\Big)m v_{cm}^2=\tfrac{7}{10}\,m v_{cm}^2=\tfrac{7}{10}\cdot3\cdot4=8.4\,\text{J}$.
\end{solutionbox}
\end{cheatproblem}
\end{KnowledgePoint}

\begin{KnowledgePoint}{Moments of Inertia \DTwo}
  \KPFormulas
  \begin{formulabox}
  \textbf{Concept description:} Moment of inertia is a measure of resistance to angular acceleration; composition rules and standard shapes allow for rapid calculation.
  
  \textbf{Core formulas:}
  \[
  \left\{\begin{aligned}
    &I=\int r^2\,dm\ (\text{axis distance }r),\\
    &\text{Parallel axis: } I=I_{\text{cm}}+Md^2,\\
    &\text{Perpendicular axis (planar lamina): } I_z=I_x+I_y.
  \end{aligned}\right.
  \]
  
  \textbf{Variable definitions:} $I_{\text{cm}}$ about COM axis; $d$ offset; $M$ total mass.
  
  
  \textbf{Prerequisites \& scope:} Perpendicular-axis requires lamina in the plane; parallel-axis requires fixed, parallel axes.
  \end{formulabox}
  
  \KPHeuristics
  \begin{heuristicsbox}
  \begin{itemize}[leftmargin=*]
    \item Decompose into standard shapes and sum moments about the same axis.
    \item Use symmetry to eliminate products of inertia; choose axes through COM when possible.
  \end{itemize}
  \vspace{0.4em}
  \begin{itemize}[leftmargin=*]
    \item Applying perpendicular-axis to 3D bodies. Fix: valid only for planar laminae.
  \end{itemize}
  \end{heuristicsbox}
  
  \KPProblems
\begin{cheatproblem}
  A thin rod of mass $m$ and length $L$ rotates about one end. Find its moment of inertia.
\begin{solutionbox}
  $I=\int_0^L x^2\,dm=\int_0^L x^2\dfrac{m}{L}\,dx=\dfrac{m}{L}\dfrac{L^3}{3}=\dfrac{mL^2}{3}$.
\end{solutionbox}
\end{cheatproblem}
\begin{cheatproblem}
  A disk of mass $M$ and radius $R$ has $I_{cm}=\tfrac12 MR^2$. Using the parallel-axis theorem, find $I$ about a point on its rim.
\begin{solutionbox}
  $I=I_{cm}+MR^2=\tfrac12 MR^2+MR^2=\tfrac32 MR^2$.
\end{solutionbox}
\end{cheatproblem}
\end{KnowledgePoint}

\begin{KnowledgePoint}{Rolling Without Slipping \DOne}
  \KPFormulas
  \begin{formulabox}
  \textbf{Concept description:} Rolling is a mixture of translation of the center of mass and rotation; the no-slip condition ties them together.

  \textbf{Core formulas:}
  \[
  \left\{\begin{aligned}
    &v_{\text{cm}}=\omega R,\\
    &K=\tfrac12 M v_{\text{cm}}^2+\tfrac12 I\omega^2.\\
  \end{aligned}\right.
  \]

  \textbf{Variable definitions:} $R$ radius; $M$ mass; $I$ moment of inertia; $v_{\text{cm}}$ center-of-mass speed.
  \end{formulabox}

  \KPHeuristics
  \begin{heuristicsbox}
  \begin{itemize}[leftmargin=*]
    \item Static friction can accelerate or decelerate rolling bodies but does no work on the body in pure rolling.
    \item Use energy for ramp problems; use $\sum \tau=I\alpha$ when forces/accelerations are requested.
  \end{itemize}
  \end{heuristicsbox}

  \KPProblems
  \begin{cheatproblem}
  A solid cylinder of mass $m$ and radius $R$ is released from rest to roll without slipping down an incline of height $h$. Find its speed at the bottom.
  \begin{solutionbox}
  Energy: $mgh=\tfrac12 m v_{cm}^2+\tfrac12 I\omega^2=\tfrac12 m v_{cm}^2+\tfrac12(\tfrac12 mR^2)(v_{cm}^2/R^2)=\tfrac12 m v_{cm}^2+\tfrac14 m v_{cm}^2=\tfrac34 m v_{cm}^2$, so $v_{cm}=\sqrt{\tfrac{4}{3}gh}$.
  \end{solutionbox}
  \end{cheatproblem}
  \begin{cheatproblem}
  A rolling sphere moves to the right and speeds up. Determine the direction of static friction on the sphere.
  \begin{solutionbox}
  For a solid sphere with $a>0$, no slip implies friction acts up the incline or in the direction that provides a torque to increase $\omega$: here, friction acts forward (to the right) at the contact point to produce a counterclockwise torque consistent with $\dot\omega>0$.
  \end{solutionbox}
  \end{cheatproblem}
\end{KnowledgePoint}

% --- New consolidated Practice Pointers for Part I ---
\subsection*{Part I: Mechanics Practice Pointers}
\addcontentsline{toc}{subsection}{Part I: Mechanics Practice Pointers}
\begin{itemize}[leftmargin=*]
  \item Physics Bowl Kinematics \& Momentum Problem 1 Page: 2
  \item Physics Bowl Circular Motion \& Energy Problem 2 Page: 3
  \item Physics Bowl Simple Machines \& Statics Problem 5 Page: 6
  \item Physics Bowl Kinematics Problem 10 Page: 11
  \item Physics Bowl Kinematics Problem 13 Page: 14
  \item Physics Bowl Work \& Energy with Friction Problem 14 Page: 14
  \item Physics Bowl Rotational Kinematics Problem 15 Page: 16
  \item Physics Bowl Rolling Dynamics Problem 16 Page: 17
  \item Physics Bowl Gravitation Problem 17 Page: 18
  \item Physics Bowl Rolling Energy Problem 20 Page: 21
  \item Physics Bowl Statics Problem 23 Page: 24
  \item Physics Bowl Relative Motion Problem 26 Page: 27
  \item Physics Bowl Circular Motion Problem 27 Page: 28
  \item Physics Bowl Work \& Energy with Friction Problem 28 Page: 28
  \item Physics Bowl Center of Mass Problem 29 Page: 29
  \item Physics Bowl Kinematics Problem 30 Page: 30
  \item Physics Bowl Momentum \& Impulse Problem 33 Page: 32
  \item Physics Bowl Kinematics Problem 35 Page: 33
  \item Physics Bowl Dynamics \& Friction Problem 36 Page: 33
  \item Physics Bowl Statics Problem 37 Page: 35
  \item Physics Bowl Rotational Energy Problem 38 Page: 36
  \item Physics Bowl Kinematics Problem 39 Page: 37
\end{itemize}

% ===================== Part II: Oscillations, Waves, Thermo/Fluids =====================
\clearpage
\parttheme{waves}
\section*{Part II: Oscillations, Waves, Thermodynamics \& Fluids}
\addcontentsline{toc}{section}{Part II: Oscillations, Waves, Thermodynamics and Fluids}

\Unit[SHM; pendulum (small-angle); wave speed $v=f\lambda$; superposition/standing waves; sound and Doppler; SHM ODE]{Unit 6: Oscillations \& Waves}

\begin{KnowledgePoint}{Simple Harmonic Motion \DOne}
  \KPFormulas
  \begin{formulabox}
  \textbf{Concept explanation:} SHM occurs when acceleration is proportional to and opposite to displacement; solutions are sinusoidal with constant amplitude (no damping).

  \textbf{Core formulas:}
  \[
  \left\{\begin{aligned}
    &x=A\cos(\omega t+\phi),\quad v=-A\omega\sin(\omega t+\phi),\quad a=-\omega^2 x,\\
    &T=\frac{2\pi}{\omega},\quad \omega=\sqrt{\frac{k}{m}}\ (\text{mass--spring}),\quad T_{\text{pend}}\approx2\pi\sqrt{\frac{\ell}{g}}\ (\text{small angle}).
  \end{aligned}\right.
  \]

  \textbf{Variable definitions:} $A$ amplitude; $\omega$ angular frequency; $\phi$ phase; $k$ spring constant; $\ell$ pendulum length.

  

  \textbf{Prerequisites \& scope:} No damping/driving; small-angle approximation for pendulum.
  \end{formulabox}

  \KPHeuristics
\begin{heuristicsbox}
  \begin{itemize}[leftmargin=*]
    \item Use energy partition $K+U=\tfrac12kA^2$ to find speeds at positions; use phase to compute time fractions.
    \item For compound oscillators, reduce to effective $k_{\text{eff}}$ or $\ell_{\text{eff}}$ before applying SHM formulas.
    \item Small-angle pendulum: check $\theta_{\max}\lesssim 10^\circ$ for $T\approx2\pi\sqrt{\ell/g}$ to be within a percent; otherwise expect longer $T$. First-order correction (radians; \DTwo result, memorize only): $\displaystyle T\approx 2\pi\sqrt{\ell/g}\,\big(1+\tfrac{\theta_0^2}{16}\big)$.
  \end{itemize}
  \vspace{0.4em}
  \begin{itemize}[leftmargin=*]
    \item Using pendulum period formula at large angles. Fix: restrict to small angles or use elliptic corrections.
  \end{itemize}
  \end{heuristicsbox}

  \KPProblems
  \begin{cheatproblem}
  A mass on a spring follows SHM with amplitude $A$ and period $T$. What fraction of the period is spent with $|x|>\tfrac{A}{2}$?
  \begin{solutionbox}
  Let $x(t)=A\cos(\omega t)$. The condition $|x(t)|>A/2$ is equivalent to $|\cos(\omega t)|>1/2$. It is simpler to calculate the fraction of time for the complementary condition, $|x(t)|\le A/2$, which corresponds to $|\cos(\omega t)|\le 1/2$.
  In one full cycle $\theta=\omega t \in [0, 2\pi)$, this holds for $\theta \in [\pi/3, 2\pi/3]$ and $\theta \in [4\pi/3, 5\pi/3]$. The total angular duration is $(\frac{2\pi}{3}-\frac{\pi}{3}) + (\frac{5\pi}{3}-\frac{4\pi}{3}) = \frac{\pi}{3}+\frac{\pi}{3}=\frac{2\pi}{3}$.
  The fraction of the period for this condition is $\frac{2\pi/3}{2\pi}=\frac{1}{3}$.
  Therefore, the fraction of the period spent with $|x|>A/2$ is $1 - 1/3 = 2/3$.
  \end{solutionbox}
  \end{cheatproblem}
\end{KnowledgePoint}

\begin{KnowledgePoint}{Waves (Traveling) \DOne}
  \KPFormulas
  \begin{formulabox}
  \textbf{Concept explanation:} Traveling waves follow $v=f\lambda$; boundary conditions set standing-wave modes; source/observer motion shifts frequency (Doppler).

  \textbf{Core formulas:}
  \[
  \left\{\begin{aligned}
    &v=f\lambda,\ \ y(x,t)=A\cos(kx-\omega t+\phi),\ k=\tfrac{2\pi}{\lambda},\ \omega=2\pi f.\\
  \end{aligned}\right.
  \]

  \textbf{Variable definitions:} $v$ wave speed in medium; $L$ length; $v_o$ observer speed; $v_s$ source speed.

  \textbf{Prerequisites \& scope:} Linear superposition; small amplitudes; Doppler formula assumes $v_o,v_s\ll v$ (nonrelativistic).
  \end{formulabox}

  \KPHeuristics
  \begin{heuristicsbox}
  \begin{itemize}[leftmargin=*]
    \item Draw mode shapes to match node/antinode boundary conditions before writing $f_n$.
    \item Use a sign diagram for Doppler to avoid sign errors; approaching increases frequency. For the formula given, this means observer towards source ($+v_o$, numerator) and source towards observer ($-v_s$, denominator). For receding, reverse these signs.
  \end{itemize}
  \vspace{0.4em}
  \begin{itemize}[leftmargin=*]
    \item Using $v$ of sound/light incorrectly across media. Fix: use the correct medium speed for $v=f\lambda$.
  \end{itemize}
  \end{heuristicsbox}

  \KPProblems
  \begin{cheatproblem}
  A wave has wavelength $\lambda=2\,\text{m}$ and frequency $f=50\,\text{Hz}$. Find its speed.
  \begin{solutionbox}
  $v=f\lambda=50\cdot2=100\,\text{m/s}$.
  \end{solutionbox}
  \end{cheatproblem}
\end{KnowledgePoint}

\begin{KnowledgePoint}{Standing Waves (Strings/Pipes) \DOne}
  \KPFormulas
  \begin{formulabox}
  \textbf{Concept explanation:} Boundaries reflect waves; interference of forward and backward waves creates nodes and antinodes with discrete mode frequencies set by geometry and boundary conditions.
  \textbf{Core formulas:}
  \[
  \left\{\begin{aligned}
    &\text{String fixed ends: } f_n=\frac{nv}{2L},\ n=1,2,\dots\\
    &\text{Open/closed pipe: }\begin{cases}
      f_n=\dfrac{nv}{2L}, & n=1,2,3,\dots\ (\text{both ends open}),\\
      f_n=\dfrac{(2n-1)v}{4L}, & n=1,2,3,\dots\ (\text{one end closed}).
    \end{cases}
  \end{aligned}\right.
  \]
  \end{formulabox}

  \KPProblems
  \begin{cheatproblem}
  A string of length $L=1.2\,\text{m}$ fixed at both ends has fundamental frequency $f_1=200\,\text{Hz}$. Find the wave speed.
  \begin{solutionbox}
  $f_1=\dfrac{v}{2L}\Rightarrow v=2Lf_1=2\cdot1.2\cdot200=480\,\text{m/s}$.
  \end{solutionbox}
  \end{cheatproblem}
\end{KnowledgePoint}

\begin{KnowledgePoint}{Doppler Effect (Fixed Medium) \DOne}
  \KPFormulas
  \begin{formulabox}
  \textbf{Concept explanation:} Relative motion between source and observer shifts the detected frequency: approaching raises $f'$ and receding lowers it; the medium is stationary.
  \textbf{Core formula:}
  \[
    f'=\frac{v\pm v_o}{v\mp v_s}\,f\ (\text{approach uses top signs}).
  \]
  \textbf{Variable definitions:} $v$ wave speed; $v_o$ observer speed; $v_s$ source speed.
  \end{formulabox}

  \KPProblems
  \begin{cheatproblem}
  An acoustic source emits $f=440\,\text{Hz}$ in air with $v=340\,\text{m/s}$. The observer moves toward the source at $v_o=10\,\text{m/s}$ while the source moves toward the observer at $v_s=20\,\text{m/s}$. Find the observed frequency.
  \begin{solutionbox}
  $f'=\dfrac{v+v_o}{v-v_s}\,f=\dfrac{350}{320}\cdot 440\approx 481\,\text{Hz}$.
  \end{solutionbox}
  \end{cheatproblem}
\end{KnowledgePoint}

\begin{KnowledgePoint}{SHM ODE and Driven Forms \DTwo}
  \KPFormulas
  \begin{formulabox}
  \textbf{Concept explanation:} The homogeneous SHM ODE has sinusoidal solutions; with driving and damping, the steady-state response depends on drive frequency and damping ratio.

  \textbf{Core formulas:}
  \[
  \left\{\begin{aligned}
    &x''+\omega_0^2 x=0 \Rightarrow x(t)=C\cos\omega_0 t+D\sin\omega_0 t,\\
    &\text{Damped: } x''+2\zeta\omega_0 x'+\omega_0^2 x=0,\ \text{under/critical/over-damped by }\zeta,\\
      &\text{Driven: } x''+2\zeta\omega_0 x'+\omega_0^2 x=\frac{F_0}{m}\cos\omega t,\\
      &\Rightarrow\ \text{amplitude peaks near }\omega\approx\omega_0,\\
      &\text{for small damping }\omega_{\text{peak}}\approx\omega_0\sqrt{1-2\zeta^2}\ (\zeta\ll1).
  \end{aligned}\right.
  \]

  \textbf{Variable definitions:} $\omega_0$ natural frequency; $\zeta$ damping ratio; $F_0$ drive amplitude.

  \textbf{Prerequisites \& scope:} Linear oscillator model; small oscillations; steady-state assumes transients have decayed.
  \end{formulabox}

  \KPHeuristics
  \begin{heuristicsbox}
  \begin{itemize}[leftmargin=*]
    \item Identify regime via $\zeta$; near resonance, estimate amplification and phase shift.
  \end{itemize}
  \vspace{0.4em}
  \begin{itemize}[leftmargin=*]
    \item Confusing natural and driving frequencies. Fix: keep $\omega_0$ (system) distinct from $\omega$ (drive).
  \end{itemize}
  \end{heuristicsbox}

  \KPProblems
\begin{cheatproblem}
  For $x''+\omega^2x=0$ with $x(0)=0$ and $\dot x(0)=v_0$, find $x(t)$ and the maximum speed.
  \begin{solutionbox}
  The general solution is $x(t)=C\cos(\omega t)+D\sin(\omega t)$.
  $x(0)=0 \Rightarrow C=0$.
  $\dot x(t) = D\omega\cos(\omega t)$, so $\dot x(0)=v_0 \Rightarrow D\omega=v_0 \Rightarrow D=v_0/\omega$.
  Thus, $x(t)=\dfrac{v_0}{\omega}\sin(\omega t)$. The velocity is $v(t) = \dot x(t) = v_0\cos(\omega t)$.
  The maximum speed is the amplitude of $v(t)$, which is $|v_0|$.
  \end{solutionbox}
  \end{cheatproblem}
\end{KnowledgePoint}

\Unit[Hydrostatics (pressure/buoyancy); continuity; Bernoulli; ideal gas; First Law and engines; entropy]{Unit 7: Fluids \& Thermodynamics}

\begin{KnowledgePoint}{Hydrostatics and Buoyancy \DOne}
  \KPFormulas
  \begin{formulabox}
  \textbf{Concept explanation:} Static fluids exert pressure that increases with depth; the buoyant force equals the weight of displaced fluid (Archimedes).

  \textbf{Core formulas:}
  \[
  \left\{\begin{aligned}
    &P=P_0+\rho g h\ (\text{hydrostatic pressure}),\\
    &F_b=\rho g V_{\text{disp}}\ (\text{buoyancy}).
  \end{aligned}\right.
  \]

  \textbf{Variable definitions:} $P_0$ reference pressure (often atmospheric at $h=0$); $\rho$ fluid density; $h$ depth; $V_{\text{disp}}$ displaced volume.

  

  \textbf{Prerequisites \& scope:} Fluid at rest (no flow), constant $\rho$ with depth (or integrate if varying); neglect surface tension unless specified.
  \end{formulabox}

  \KPHeuristics
  \begin{heuristicsbox}
  \begin{itemize}[leftmargin=*]
    \item Draw free-body diagrams (FBD) of floating/sinking bodies: set $F_b$ vs weight vs any tension to solve equilibrium.
    \item Choose a definite reference level for $h$ and keep $P_0$ the same when comparing points.
    \item Submerged fraction for floating: $\dfrac{V_{sub}}{V}=\dfrac{\rho_b}{\rho}$ (with $0<\rho_b\le\rho$).
    \item If $\rho_b>\rho$ and the object is released freely, the initial net force is downward: $mg-F_b>0$.
    \item If later supported (bottom contact or tension), static equilibrium requires $T+F_b=mg$.
    \item Gauge vs absolute pressure: $\Delta P=\rho g h$ is a \emph{gauge} difference; absolute pressure is $P= P_{\text{atm}}+\rho g h$ when the surface is open to atmosphere.
  \end{itemize}
  \vspace{0.4em}
  \begin{itemize}[leftmargin=*]
    \item Using object's volume instead of displaced volume for $F_b$. Fix: use actual displaced fluid volume (submerged part only).
  \end{itemize}
  \end{heuristicsbox}

  \KPProblems
\begin{cheatproblem}
  A block of volume $V$ and density $\rho_b$ floats in a liquid of density $\rho$. What fraction of its volume is submerged?
  \begin{solutionbox}
  At equilibrium $\rho g V_{sub}=\rho_b g V$, so $V_{sub}/V=\rho_b/\rho$.
  \end{solutionbox}
  \end{cheatproblem}
% (duplicate problem removed)
\end{KnowledgePoint}

\begin{KnowledgePoint}{Continuity and Bernoulli \DOne}
  \KPFormulas
  \begin{formulabox}
  \textbf{Concept explanation:} In steady incompressible flow, mass conservation gives $Av=\text{const}$; along a streamline with negligible viscosity, mechanical energy per volume is constant (Bernoulli).

  \textbf{Core formulas:}
  \[
  \left\{\begin{aligned}
    &\text{Continuity: } A_1 v_1=A_2 v_2\ (\rho \ \text{constant}).\\
    &\text{Bernoulli: } P+\tfrac12\rho v^2+\rho g y=\text{const}\ (\text{along a streamline}).
  \end{aligned}\right.
  \]

  \textbf{Variable definitions:} $A$ cross-sectional area; $v$ speed; $P$ pressure; $y$ elevation; $\rho$ density.

  

  \textbf{Prerequisites \& scope:} Steady, incompressible, non-viscous flow; apply Bernoulli along a streamline, not across shocks or with pumps/turbines unaccounted.
  \end{formulabox}

  \KPHeuristics
  \begin{heuristicsbox}
  \begin{itemize}[leftmargin=*]
    \item Check assumptions (steady/incompressible/irrotational) before using Bernoulli; otherwise use energy loss terms.
    \item Combine continuity with Bernoulli to eliminate speeds or pressures efficiently.
  \item Use stagnation points: where $v=0$, the total (stagnation) pressure is $P_0=P+\tfrac12\rho v^2$ upstream along a streamline.
  \end{itemize}
  \vspace{0.4em}
  \begin{itemize}[leftmargin=*]
    \item Using Bernoulli across different streamlines where viscous losses or pumps exist. Fix: apply along a single streamline and include head gains/losses when needed.
  \end{itemize}
  \end{heuristicsbox}

  \KPProblems
\begin{cheatproblem}
  Water flows through a horizontal pipe from diameter $D_1=0.1\,\text{m}$ to $D_2=0.05\,\text{m}$. If $v_1=2\,\text{m/s}$, find $v_2$ using continuity.
\begin{solutionbox}
  $A_1v_1=A_2v_2\Rightarrow\pi(D_1/2)^2v_1=\pi(D_2/2)^2v_2\Rightarrow v_2=v_1\dfrac{D_1^2}{D_2^2}=2\dfrac{0.01}{0.0025}=8\,\text{m/s}$.
\end{solutionbox}
\end{cheatproblem}
\begin{cheatproblem}
  At point 1 in a pipe, $P_1=1.0\times10^5\,\text{Pa}$, $v_1=2\,\text{m/s}$, $y_1=0$. At point 2, $v_2=5\,\text{m/s}$, $y_2=3\,\text{m}$. Find $P_2$ for water ($\rho=1000\,\text{kg/m}^3$).
\begin{solutionbox}
  Bernoulli: $P_1+\tfrac12\rho v_1^2+\rho gy_1=P_2+\tfrac12\rho v_2^2+\rho gy_2$.
  Compute in SI with scientific notation:
  $\tfrac12\rho(v_1^2-v_2^2)=0.5\times10^3\,(4-25)=-1.05\times10^4\,\text{Pa}$ and
  $\rho g(y_2-y_1)=10^3\times9.8\times3=2.94\times10^4\,\text{Pa}$.
  Hence $P_2=1.00\times10^5-1.05\times10^4-2.94\times10^4=6.01\times10^4\,\text{Pa}$.
\end{solutionbox}
\end{cheatproblem}
\end{KnowledgePoint}

\begin{KnowledgePoint}{Ideal Gas and First Law \DOne}
  \KPFormulas
  \begin{formulabox}
  \textbf{Concept explanation:} $PV=nRT$ holds for ideal gases; the First Law links changes in internal energy to heat and work with clear sign conventions.

  \textbf{Core formulas:}
  \[
  \left\{\begin{aligned}
      &PV=nRT,\quad U=\tfrac{f}{2}nRT,\\
      &\text{where } f\ \text{is dof (e.g., }f=3\ \text{monatomic},\ 5\ \text{diatomic at room T)},\\
    &\Delta U=Q+W_{\text{on}}\ \ (W_{\text{on}}=\text{work done on the system}),\\
    &W_{\text{on}}=-\int P\,dV\quad (\text{so }W_{\text{by}}\equiv-\,W_{\text{on}}=\int P\,dV).
  \end{aligned}\right.
  \]

  \textbf{Variable definitions:} $P,V,T$ pressure/volume/temperature; $n$ moles; $R$ gas constant; $Q$ heat into system; $W_{\text{on}}$ work on system; $W_{\text{by}}$ work done by the gas; $U$ internal energy.

  

  \textbf{Prerequisites \& scope:} Ideal gas approximation; $U$ depends only on $T$ for ideal gases; sign convention must be consistent.
  \end{formulabox}

  \KPHeuristics
  \begin{heuristicsbox}
  \begin{itemize}[leftmargin=*]
    \item Identify process (isochoric/isobaric/isothermal/adiabatic) to pick $W,\,Q,\,\Delta U$ quickly.
    \item Draw $P$–$V$ diagrams: areas give work; direction indicates sign.
  \end{itemize}
  \vspace{0.4em}
  \begin{itemize}[leftmargin=*]
    \item Mixing sign conventions for work. Fix: adopt $\Delta U=Q+W_{\text{on}}$ (work on system positive) consistently; then $W_{\text{by}}=-W_{\text{on}}$.
  \end{itemize}
  \end{heuristicsbox}

  \KPProblems
\begin{cheatproblem}
  An ideal gas undergoes an isothermal expansion from volume $V_1$ to $V_2$ at temperature $T$. Find the work done by the gas.
  \begin{solutionbox}
  $W=\int_{V_1}^{V_2} \frac{nRT}{V}\,dV=nRT\ln\!\frac{V_2}{V_1}$.
  \end{solutionbox}
  \end{cheatproblem}
\begin{cheatproblem}
  An ideal gas at $P=2\times10^5\,\text{Pa}$ and $V=0.01\,\text{m}^3$ has $n=1\,\text{mol}$. Find the temperature $T$ using $PV=nRT$ with $R=8.314\,\text{J/(mol·K)}$.
\begin{solutionbox}
  $T=\dfrac{PV}{nR}=\dfrac{2\times10^5\cdot0.01}{1\cdot8.314}\approx240\,\text{K}$.
  \end{solutionbox}
  \end{cheatproblem}
\end{KnowledgePoint}

\begin{KnowledgePoint}{Entropy and Carnot \DOne}
  \KPFormulas
  \begin{formulabox}
  \textbf{Concept explanation:} Entropy quantifies thermal disorder and increases in irreversible processes; Carnot gives the maximum efficiency of heat engines between two reservoirs.

  \textbf{Core formulas:}
  \[
  \left\{\begin{aligned}
    &\Delta S=\int_{\text{rev}} \frac{\delta Q}{T},\\
    &\text{Carnot bound: } \eta_{\max}=1-\frac{T_c}{T_h}.
  \end{aligned}\right.
  \]

  \textbf{Variable definitions:} $S$ entropy; $T_h,T_c$ hot/cold absolute temperatures; $\delta Q$ infinitesimal heat (reversible path).

  

  \textbf{Prerequisites \& scope:} Absolute temperatures (Kelvin); reversible paths for definition; real engines achieve less than Carnot due to irreversibilities.
  \end{formulabox}

  \KPHeuristics
  \begin{heuristicsbox}
  \begin{itemize}[leftmargin=*]
    \item Compute $\Delta S$ along a convenient reversible path (e.g., isothermal + isochoric steps).
    \item For engine limits, compare cycle temperatures to $T_h,T_c$ to bound $\eta$ quickly.
  \end{itemize}
  \vspace{0.4em}
  \begin{itemize}[leftmargin=*]
    \item Using Celsius in $\eta=1-T_c/T_h$. Fix: convert to Kelvin.
  \end{itemize}
  \end{heuristicsbox}

  \KPProblems
\begin{cheatproblem}
  A Carnot engine operates between $T_h=500\,\text{K}$ and $T_c=300\,\text{K}$. Find its maximum efficiency.
\begin{solutionbox}
  $\eta_{\max}=1-\dfrac{T_c}{T_h}=1-\dfrac{300}{500}=0.4=40\%$.
\end{solutionbox}
\end{cheatproblem}
\begin{cheatproblem}
  An ideal gas expands reversibly at constant $T=400\,\text{K}$ from $V_1=1\,\text{m}^3$ to $V_2=2\,\text{m}^3$. Find the entropy change if $n=1\,\text{mol}$.
\begin{solutionbox}
  $\Delta S=\dfrac{Q_{rev}}{T}=\dfrac{nRT\ln(V_2/V_1)}{T}=nR\ln2=8.314\ln2\approx5.76\,\text{J/K}$.
\end{solutionbox}
\end{cheatproblem}
\end{KnowledgePoint}

\begin{KnowledgePoint}{Heat Engines and Efficiency \DOne}
  \KPFormulas
  \begin{formulabox}
  \textbf{Core formulas:}
  \[
  \left\{\begin{aligned}
    &W_{\text{by}}=\int P\,dV,\quad W_{\text{on}}=-W_{\text{by}},\quad \eta=\frac{W_{\text{by}}}{Q_h},\\
    &\eta_{\max,\,\text{Carnot}}=1-\frac{T_c}{T_h}.
  \end{aligned}\right.
  \]
  \textbf{Variable definitions:} $Q_h$ heat absorbed from hot reservoir; $T_h,T_c$ absolute temperatures of hot/cold reservoirs; $W_{\text{by}}$ work done by the gas.
  \end{formulabox}
\end{KnowledgePoint}

% --- New consolidated Practice Pointers for Part II ---
\subsection*{Part II: Oscillations, Waves, Thermodynamics and Fluids Practice Pointers}
\addcontentsline{toc}{subsection}{Part II: Oscillations, Waves, Thermodynamics and Fluids Practice Pointers}
\begin{itemize}[leftmargin=*]
  \item Physics Bowl Waves \& Sound Problem 3 Page: 4
  \item Physics Bowl Thermodynamics \& Phase Change Problem 4 Page: 5
  \item Physics Bowl Thermodynamics \& Phase Equilibrium Problem 6 Page: 7
  \item Physics Bowl Fluid Mechanics Problem 11 Page: 12
  \item Physics Bowl Fluid Mechanics \& Projectile Motion Problem 18 Page: 19
  \item Physics Bowl Thermodynamics \& Engines Problem 19 Page: 20
  \item Physics Bowl Oscillations Problem 24 Page: 25
  \item Physics Bowl Oscillations Problem 31 Page: 31
\end{itemize}

% ===================== Part III: Electricity & Magnetism =====================
\clearpage
\parttheme{em}
\section*{Part III: Electricity \texorpdfstring{\&}{&} Magnetism}
\addcontentsline{toc}{section}{Part III: Electricity \& Magnetism}

\Unit[Coulomb force; electric field/potential; capacitors and energy; Gauss's law]{Unit 8: Electrostatics}

\begin{KnowledgePoint}{Coulomb, Field and Potential \DOne}
  \KPFormulas
  \begin{formulabox}
  \textbf{Concept explanation:} Point charges interact by an inverse-square law; electric field and potential describe force per unit charge and energy per unit charge.

  \textbf{Core formulas:}
  \[
  \left\{\begin{aligned}
    &|\vec F|=k\,\frac{|q_1 q_2|}{r^2}\ (\text{along the line of centers}),\\
    &\vec E=\frac{\vec F}{q},\quad V=\frac{U}{q},\quad \Delta U=-q\int \vec E\cdot d\vec r,\\
    &\text{Point charge: } E=\frac{1}{4\pi\varepsilon_0}\frac{Q}{r^2},\ V=\frac{1}{4\pi\varepsilon_0}\frac{Q}{r}.
  \end{aligned}\right.
  \]

  \textbf{Variable definitions:} $q,Q$ charges; $r$ separation; $k=1/(4\pi\varepsilon_0)$; $U$ potential energy.

  \textbf{Prerequisites \& scope:} Electrostatics (charges at rest); superposition holds; signs determine directions.
  \end{formulabox}

  \KPHeuristics
  \begin{heuristicsbox}
  \begin{itemize}[leftmargin=*]
    \item Apply symmetry (dipoles, rings, infinite sheets) to cancel parts before integrating.
    \item Apply potential for conservative additions first, then differentiate to get fields.
  \end{itemize}
  \vspace{0.4em}
  \begin{itemize}[leftmargin=*]
    \item Not remembering vector directions for $\vec E$ and $\vec F$. Fix: graph direction first, calculate magnitude second.
  \end{itemize}
  \end{heuristicsbox}

  \KPProblems
\begin{cheatproblem}
  Two point charges $+Q$ are at $(\pm a,0)$. Find the electric field on the $y$-axis at $(0,y)$.
  \begin{solutionbox}
  Horizontal components cancel; vertical add: $E_y=2\,kQ\,\dfrac{y}{(a^2+y^2)^{3/2}}$.
  \end{solutionbox}
  \end{cheatproblem}
\end{KnowledgePoint}

% Constants and units quick reference (placed near EM section for convenience)
\begin{insightbox}
\textbf{Constants/units:} $\varepsilon_0=8.85\times10^{-12}\,\mathrm{F/m}$, $\mu_0=4\pi\times10^{-7}\,\mathrm{H/m}$, $c=3.00\times10^8\,\mathrm{m/s}$, $e=1.60\times10^{-19}\,\mathrm{C}$. Use SI unless specified.
\end{insightbox}

% (Removed redundant KnowledgePoint: Point Charge Field; content covered in Unit 8 Electrostatics)

\begin{KnowledgePoint}{Capacitors and Energy \DOne}
  \KPFormulas
  \begin{formulabox}
  \textbf{Concept explanation:} A capacitor is a device that holds equal and opposite charge on two conductors with a gap or dielectric between them. The stored charge at a given potential difference will be proportional to the geometry and material. Networks simplify by simple series/parallel formulas and energy can be traced by $U=\tfrac12 C V^2$.

  \textbf{Core formulas:}
  \[
  \left\{\begin{aligned}
    &C=\varepsilon_0\,\frac{A}{d}\ \text{(parallel plates in vacuum)},\quad C=\varepsilon_r\varepsilon_0\,\frac{A}{d}\ \text{(uniform dielectric)},\\
    &\text{Series: } \frac{1}{C_s}=\sum \frac{1}{C_i},\quad \text{Parallel: } C_p=\sum C_i,\\
    &U=\tfrac12 C V^2=\tfrac12 QV=\frac{Q^2}{2C}.
  \end{aligned}\right.
  \]

  \textbf{Variable definitions:} $A$ plate area; $d$ separation; $\varepsilon_0$ vacuum permittivity; $\varepsilon_r$ relative permittivity; $Q$ charge; $V$ voltage.

  \textbf{Prerequisites \& scope:} Edge effects neglected; linear dielectrics; use equivalent capacitance to reduce networks.
  \end{formulabox}

  \KPHeuristics
  \begin{heuristicsbox}
  \begin{itemize}[leftmargin=*]
    \item Parallel by series and symmetry simplify before writing node/loop equations.
    \item Use $U=\tfrac12 CV^2$ to compare energy storage or redistribution after reconfiguration.
  \end{itemize}
  \vspace{0.4em}
  \begin{itemize}[leftmargin=*]
    \item Assuming charge conservation on each plate when switches change connectivity. Fix: conserve charge on isolated conductors only.
  \end{itemize}
  \end{heuristicsbox}

  \KPProblems
\begin{cheatproblem}
  Two capacitors $C_1=2\,\mu\text{F}$ and $C_2=3\,\mu\text{F}$ are in series. Find the equivalent capacitance.
\begin{solutionbox}
  $\dfrac{1}{C_s}=\dfrac{1}{C_1}+\dfrac{1}{C_2}=\dfrac{1}{2}+\dfrac{1}{3}=\dfrac{5}{6}\Rightarrow C_s=\dfrac{6}{5}=1.2\,\mu\text{F}$.
\end{solutionbox}
\end{cheatproblem}
\begin{cheatproblem}
  A capacitor with $C=10\,\mu\text{F}$ is charged to $V=50\,\text{V}$. Find the energy stored.
\begin{solutionbox}
  $U=\tfrac12 CV^2=\tfrac12\cdot10\times10^{-6}\cdot2500=12.5\times10^{-3}=12.5\,\text{mJ}$.
\end{solutionbox}
\end{cheatproblem}
\end{KnowledgePoint}

\begin{KnowledgePoint}{Gauss's Law \DOne}
  \KPFormulas
  \begin{formulabox}
  \textbf{Concept explanation:} The flux of $\vec E$ through a closed surface equals enclosed charge over $\varepsilon_0$; symmetry lets you get fields without integration.

  \textbf{Core formula:}
  \[
  \oint \vec E\cdot d\vec A=\frac{Q_{enc}}{\varepsilon_0}.
  \]

  \textbf{Variable definitions:} $Q_{enc}$ charge enclosed; $d\vec A$ outward area element; choose Gaussian surface aligned to symmetry.

  

  \textbf{Prerequisites \& scope:} Use for infinite planes/cylinders/spheres; for conductors, $E=0$ inside and charges reside on surfaces. Within uniform dielectrics/insulators with embedded charge, fields may exist inside the material (i.e., $E$ need not vanish). \textit{Under \DOne: memorize the integral statement; derivations are not required.}
  \end{formulabox}

  \KPHeuristics
  \begin{heuristicsbox}
  \begin{itemize}[leftmargin=*]
    \item Pick surfaces where $E$ is constant and parallel to $d\vec A$ over large patches (sphere/cylinder/plane).
    \item For conductors in electrostatics, set $E_{\text{inside}}=0$ and use boundary conditions for surface charge; for dielectrics, prefer symmetry and superposition without advanced $\vec D$ formalism.
  \end{itemize}
  \vspace{0.4em}
  \begin{itemize}[leftmargin=*]
    \item Choosing a Gaussian surface that doesn't match symmetry, forcing difficult integrals. Fix: reselect surface to exploit symmetry.
  \end{itemize}
  \end{heuristicsbox}

  \KPProblems
\begin{cheatproblem}
  Using Gauss's law, find $E(r)$ outside a uniformly charged sphere of radius $R$ and total charge $Q$.
  \begin{solutionbox}
  Gaussian sphere: $E\cdot4\pi r^2=Q/\varepsilon_0$, so $E(r)=\dfrac{1}{4\pi\varepsilon_0}\dfrac{Q}{r^2}$ for $r\ge R$.
  \end{solutionbox}
  \end{cheatproblem}
\end{KnowledgePoint}

% (Removed outdated Practice Pointers for Part III Unit 8)

\Unit[Ohm's law and power; series/parallel reductions; Kirchhoff (KCL/KVL); RC qualitative; RC exact]{Unit 9: DC Circuits}

\begin{KnowledgePoint}{Ohm's Law and Reductions \DOne}
  \KPFormulas
  \begin{formulabox}
  \textbf{Concept explanation:} Ohm's law relates voltage, current, and resistance; power forms help rank dissipation; series/parallel laws simplify networks.

  \textbf{Core formulas:}
  \[
  \left\{\begin{aligned}
    &V=IR,\quad P=IV=I^2R=\frac{V^2}{R},\\
    &R_s=\sum R_i,\quad \frac{1}{R_p}=\sum \frac{1}{R_i}.
  \end{aligned}\right.
  \]

  \textbf{Variable definitions:} $V$ voltage; $I$ current; $R$ resistance; $P$ power.

  \textbf{Prerequisites \& scope:} Ohmic elements only; temperature dependence ignored unless specified.
  \end{formulabox}

  \KPHeuristics
  \begin{heuristicsbox}
\begin{itemize}[leftmargin=*]
  \item Reduce networks with series/parallel and symmetry; then solve KCL/KVL for the rest of unknowns alone.
  \item Equate power by using $I^2 R$ or $V^2/R$ based on conditions of fixed current/voltage.
  \item In bridge-type circuits, look for equal potentials in a branch (through symmetry or KCL) to remove it.
\end{itemize}
  \vspace{0.4em}
  \begin{itemize}[leftmargin=*]
    \item Mixing fixed-voltage and fixed-current contexts when comparing brightness. Fix: pick the power form consistent with constraints.
  \end{itemize}
  \end{heuristicsbox}

  \KPProblems
\begin{cheatproblem}
  Three resistors of $R$ are in parallel; find the equivalent resistance.
  \begin{solutionbox}
  $\dfrac{1}{R_{eq}}=\dfrac{1}{R}+\dfrac{1}{R}+\dfrac{1}{R}$ so $R_{eq}=R/3$.
  \end{solutionbox}
  \end{cheatproblem}
\end{KnowledgePoint}

\begin{KnowledgePoint}{Kirchhoff Laws (KCL/KVL) \DOne}
  \KPFormulas
  \begin{formulabox}
  \textbf{Concept explanation:} Kirchhoff's current and voltage laws enforce charge and energy conservation; first-order RC circuits charge and discharge exponentially with time constant $\tau$.

  \textbf{Core formulas:}
  \[
  \left\{\begin{aligned}
    &\text{KCL: } \sum I_{\text{in}}=\sum I_{\text{out}}\ (\text{at a node}),\\
    &\text{KVL: } \sum \Delta V=0\ (\text{around a loop}),\\
    &\text{Charging (step from }V_0\to V):\\
    &\quad V_C(t)=V+\big(V_0-V\big)e^{-t/RC},\\
    &\quad I(t)=\frac{V-V_0}{R}\,e^{-t/RC},\\
    &\text{Discharging (to 0): }\\
    &\quad V_C(t)=V_0\,e^{-t/RC},\\
    &\quad I(t)=-\frac{V_0}{R}e^{-t/RC},\\
      &\text{General: }\ V_C(t)=V_\infty+\big(V_0-V_\infty\big)e^{-t/\tau}.
  \end{aligned}\right.
  \]

  \textbf{Variable definitions:} $\tau=RC$ time constant; $V$ source voltage; $V_C$ capacitor voltage; $I$ branch current.

  \textbf{Prerequisites \& scope:} Linear time-invariant components; piecewise-constant sources for standard transients.
  \end{formulabox}

  \KPHeuristics
  \begin{heuristicsbox}
\begin{itemize}[leftmargin=*]
  \item Use series/parallel reductions or source transformations conceptually when helpful, but solve RC timing with baseline KCL/KVL forms.
  \item Check for limiting values at $t=0^+$ and $t=\infty$ to test expressions; impose continuity of $V_C$ at switching times.
  \item Time constant: $\tau=RC$ for the basic first-order RC considered here.
\end{itemize}
  \vspace{0.4em}
  \begin{itemize}[leftmargin=*]
    \item Letting capacitor voltage jump at $t=0$. Fix: enforce continuity of $V_C$ and initial condition from prior steady state.
  \end{itemize}
  \end{heuristicsbox}

  \KPProblems
  \begin{cheatproblem}
  An $RC$ circuit with $V$ applied at $t=0$ has $R=2\,\Omega$, $C=1\,\text{F}$. Find $V_C(t)$.
  \begin{solutionbox}
  $V_C(t)=V(1-e^{-t/RC})=V(1-e^{-t/2})$.
  \end{solutionbox}
  \end{cheatproblem}
\begin{cheatproblem}
  Quick check for limits: For any first-order $RC$ with step to $V_\infty$, verify $V_C(0^+)=V_C(0^-)$ (no jump) and $V_C(\infty)=V_\infty$. With $V_C(0^-)=V_0$, the standard form $V_C(t)=V_\infty+(V_0-V_\infty)e^{-t/\tau}$ satisfies both.
  \begin{solutionbox}
    Evaluate: $V_C(0^+)=V_\infty+(V_0-V_\infty)=V_0$ (continuous). As $t\to\infty$, $e^{-t/\tau}\to0$, so $V_C\to V_\infty$.
  \end{solutionbox}
\end{cheatproblem}
\end{KnowledgePoint}

\begin{KnowledgePoint}{RC Transients (First Order) \DOne}
  \KPFormulas
  \begin{formulabox}
  \textbf{Core results:}
  \[
  \left\{\begin{aligned}
    &\tau=RC,\\
    &\text{Charge to }V:\ V_C(t)=V+\big(V_0-V\big)e^{-t/RC},\\
    &\text{Discharge to 0: } V_C(t)=V_0 e^{-t/RC}.
  \end{aligned}\right.
  \]
  \end{formulabox}

  \KPProblems
  \begin{cheatproblem}
  An $RC$ circuit with $V$ applied at $t=0$ has $R=2\,\Omega$, $C=1\,\text{F}$. Find $V_C(t)$.
  \begin{solutionbox}
  $V_C(t)=V(1-e^{-t/RC})=V(1-e^{-t/2}).$
  \end{solutionbox}
  \end{cheatproblem}
\end{KnowledgePoint}

% (Removed outdated Practice Pointers for Part III Unit 9)

\Unit[Lorentz force (charges/wires); Faraday-Lenz induction; EM spectrum; Ampere law]{Unit 10: Magnetism \& Induction}

\begin{KnowledgePoint}{Lorentz Force \DOne}
  \KPFormulas
  \begin{formulabox}
  \textbf{Concept explanation:} A moving charge feels $q\vec E$ and $q\,\vec v\times\vec B$; the magnetic force stays perpendicular to velocity, so it deflects direction without changing speed.

  \textbf{Core formulas:}
  \[
  \left\{\begin{aligned}
    &\vec F=q\,(\vec E+\vec v\times\vec B)
  \end{aligned}\right.
  \]

  \textbf{Variable definitions:} $q$ charge; $\vec v$ particle velocity; $\vec E,\vec B$ fields.

\textbf{Prerequisites \& scope:} Nonrelativistic; right-hand rule for cross products.
  \end{formulabox}

  \KPHeuristics
  \begin{heuristicsbox}
  \begin{itemize}[leftmargin=*]
    \item Use right-hand rule consistently; reverse direction for negative charges.
    \item Magnetic force does no work (always perpendicular to $\vec v$), so magnetic fields alone cannot change particle speed.
  \end{itemize}
  \vspace{0.4em}
  \begin{itemize}[leftmargin=*]
    \item Using $q$'s sign incorrectly in $q\,\vec v\times\vec B$. Fix: compute direction for positive charge, then flip if $q<0$.
  \end{itemize}
  \end{heuristicsbox}

  \KPProblems
  \begin{cheatproblem}
  A particle of charge $q$ enters a uniform magnetic field $\vec B$ perpendicular to its velocity with speed $v$. Find the radius and period of its circular motion (neglect $\vec E$).
  \begin{solutionbox}
  Magnetic force provides centripetal: $qvB=\dfrac{mv^2}{r}\Rightarrow r=\dfrac{mv}{qB}$. The period is $T=\dfrac{2\pi r}{v}=\dfrac{2\pi m}{qB}$.
  \end{solutionbox}
  \end{cheatproblem}
\end{KnowledgePoint}

\begin{KnowledgePoint}{Magnetic Force on Wires \DOne}
  \KPFormulas
  \begin{formulabox}
  \textbf{Core formulas:}
  \[
    \vec F=I\,\vec L\times\vec B,\quad |\vec F|=ILB\sin\theta.
  \]
  \textbf{Variable definitions:} $I$ current; $\vec L$ directed along the current segment with magnitude $L$; $\vec B$ magnetic field; $\theta$ angle between $\vec L$ and $\vec B$.
  \end{formulabox}

  \KPHeuristics
  \begin{heuristicsbox}
  \begin{itemize}[leftmargin=*]
    \item For loops, integrate $d\vec F=I\,d\vec l\times\vec B$ and exploit symmetry.
  \end{itemize}
  \end{heuristicsbox}

  \KPProblems
  \begin{cheatproblem}
  A wire of length $L$ carries current $I$ in a uniform field $\vec B$ perpendicular to the wire. Find the magnitude of magnetic force.
  \begin{solutionbox}
  $F=ILB$.
  \end{solutionbox}
  \end{cheatproblem}
\end{KnowledgePoint}

\begin{KnowledgePoint}{Faraday-Lenz \DOne}
  \KPFormulas
\begin{formulabox}
  \textbf{Concept explanation:} Changing magnetic flux induces an emf that opposes the change (Lenz); steady currents set magnetic fields constrained by Amp\`ere's law.

  \textbf{Core formulas:}
  \[
  \left\{\begin{aligned}
    &\mathcal E=\oint_{\partial S}\vec E\cdot d\vec l=-\frac{d\Phi_B}{dt},\ \ \Phi_B=\iint_S \vec B\cdot d\vec A,\ \ \Phi_E=\iint_S \vec E\cdot d\vec A,\\
    &\text{(see Amp\`ere/Maxwell--Amp\`ere for magnetic circulation)}
  \end{aligned}\right.
  \]

  \textbf{Variable definitions:} $\mathcal E$ induced emf (scalar, $\mathcal E=\oint_{\partial S}\vec E\cdot d\vec l$); $\Phi_B=\iint_S \vec B\cdot d\vec A$ magnetic flux; $\Phi_E=\iint_S \vec E\cdot d\vec A$ electric flux; $I_{enc}$ enclosed current.
  
  \textbf{Prerequisites \& scope:} Under \DOne memorize the integral form (no derivation). Generally assume quasi-static fields; for time-varying fields use Maxwell--Amp\`ere with displacement current.
  \end{formulabox}

  \begin{insightbox}
  \textbf{Terminology note}: \emph{Electric circulation} refers to the line integral of the electric field that defines emf, $\mathcal E=\oint_{\partial S}\vec E\cdot d\vec l$. \emph{Magnetic circulation} refers to $\oint\vec B\cdot d\vec l$ as used in Amp\`ere/Maxwell--Amp\`ere. These are distinct: $\mathcal E$ is a scalar (emf), while $\vec E$ and $\vec B$ are fields.
  \end{insightbox}

  \KPHeuristics
  \begin{heuristicsbox}
  \begin{itemize}[leftmargin=*]
    \item Sketch the loop and determine the positive normal; apply Lenz's rule to deduce the direction of the induced current.
    \item Apply circular/rectangular Amperian loops along symmetry for infinite wires/solenoids.
  \end{itemize}
  \vspace{0.4em}
  \begin{itemize}[leftmargin=*]
    \item Missing displacement current for charging capacitors. Fix: include $\varepsilon_0\,d\Phi_E/dt$ in Maxwell–Amp\`ere when fields vary.
  \end{itemize}
  \end{heuristicsbox}

  \KPProblems
  \begin{cheatproblem}
  In a loop of area $A$, the magnetic field increases as $B(t)=B_0+kt$. Find the induced emf.
  \begin{solutionbox}
  $\mathcal E=\left|\dfrac{d\Phi}{dt}\right|=\left|\dfrac{d(BA)}{dt}\right|=kA$.
  \end{solutionbox}
  \end{cheatproblem}
\end{KnowledgePoint}

\begin{KnowledgePoint}{Ampere and Maxwell--Ampere \DTwo}
  \KPFormulas
  \begin{formulabox}
  \textbf{Core formulas:}
  \[
  \left\{\begin{aligned}
    &\oint \vec B\cdot d\vec l=\mu_0 I_{enc}\ (\text{steady currents}),\\
    &\text{Maxwell--Ampere (general): }\oint \vec B\cdot d\vec l=\mu_0 I_{enc}+\mu_0\varepsilon_0\,\frac{d\Phi_E}{dt}.
  \end{aligned}\right.
  \]
  \end{formulabox}
\end{KnowledgePoint}

\begin{KnowledgePoint}{EM Spectrum and Maxwell (Concept) \DOne}
  \KPFormulas
  \begin{formulabox}
  \textbf{Concept explanation:} Electromagnetic waves range from radio to gamma; Maxwell's equations couple $\vec E$ and $\vec B$ and give wave speed $c$ in vacuum.

  \textbf{Formulas \& Concepts:}
  \[
  \left\{\begin{aligned}
    &c=\frac{1}{\sqrt{\mu_0\varepsilon_0}},\\
      &\text{Spectrum ordering by frequency: radio }\to\ \text{microwave }\to\ \text{IR }\to\ \text{visible},\\
      &\text{then }\ \text{UV }\to\ \text{X }\to\ \text{gamma}.
  \end{aligned}\right.
  \]

  \textbf{Prerequisites \& scope:} Vacuum relations shown; material dispersion alters speed and wavelength.
  \end{formulabox}

  \KPHeuristics
  \begin{heuristicsbox}
  \begin{itemize}[leftmargin=*]
    \item   Recall typical sources: antennas (radio), thermal (IR), electronic transitions (visible/UV), inner-shell transitions (X/gamma).
    \item Use $c=f\lambda$ with medium refractive index $n$ via $v=c/n$.
  \end{itemize}
  \end{heuristicsbox}

  \KPProblems
\begin{cheatproblem}
  Light in vacuum has wavelength $\lambda=600\,\text{nm}$ and speed $c=3\times10^8\,\text{m/s}$. Find its frequency.
\begin{solutionbox}
  $f=\dfrac{c}{\lambda}=\dfrac{3\times10^8}{600\times10^{-9}}=5\times10^{14}\,\text{Hz}$.
\end{solutionbox}
\end{cheatproblem}
\begin{cheatproblem}
  Rank the following by increasing photon energy: radio, visible, X-ray.
\begin{solutionbox}
  Higher frequency means higher photon energy $E=hf$. Ordering: radio $<$ visible $<$ X-ray.
\end{solutionbox}
\end{cheatproblem}
\end{KnowledgePoint}

% --- New consolidated Practice Pointers for Part III ---
\subsection*{Part III: Electricity \texorpdfstring{\&}{&} Magnetism Practice Pointers}
\addcontentsline{toc}{subsection}{Part III: Electricity \& Magnetism Practice Pointers}
\begin{itemize}[leftmargin=*]
  \item Physics Bowl DC Circuits Problem 7 Page: 8
  \item Physics Bowl Electricity \& Magnetism Problem 8 Page: 9
  \item Physics Bowl DC Circuits Problem 9 Page: 10
  \item Physics Bowl RC Circuits Problem 12 Page: 13
  \item Physics Bowl DC Circuits Problem 22 Page: 23
\end{itemize}

% ===================== Part IV: Optics & Modern Physics =====================
\clearpage
\parttheme{optics}
\section*{Part IV: Optics \texorpdfstring{\&}{&} Modern Physics}
\addcontentsline{toc}{section}{Part IV: Optics and Modern Physics}

\Unit[Reflection; refraction (Snell); thin lens and magnification; interference/diffraction]{Unit 11: Optics}

\begin{KnowledgePoint}{Reflection and Refraction \DOne}
  \KPFormulas
  \begin{formulabox}
  \textbf{Concept explanation:} Light reflects with equal incident and reflected angles; refraction across media obeys Snell's law.

  \textbf{Core formulas:}
  \[
  \left\{\begin{aligned}
    &\theta_i=\theta_r,\\
    &n_1\sin\theta_1=n_2\sin\theta_2,\quad \text{TIR when }\theta_1>\theta_c=\arcsin(n_2/n_1)\ (n_1>n_2).
  \end{aligned}\right.
  \]

  \textbf{Variable definitions:} $n$ refractive index; $\theta$ angles measured to the normal.

  \textbf{Prerequisites \& scope:} Geometric optics regime; isotropic media; polarization effects ignored here.
  \end{formulabox}

  \KPHeuristics
  \begin{heuristicsbox}
  \begin{itemize}[leftmargin=*]
    \item Draw the normal and principal rays first; search for total internal reflection when going to a lower-$n$ medium.
    \item Use reversibility of light to validate constructions.
  \end{itemize}
  \end{heuristicsbox}

  \KPProblems
\begin{cheatproblem}
  Light travels from air ($n_1=1$) into water ($n_2=1.33$) at incidence angle $\theta_1=40^\circ$. Find the refraction angle $\theta_2$.
\begin{solutionbox}
  Snell: $n_1\sin\theta_1=n_2\sin\theta_2\Rightarrow\sin\theta_2=\dfrac{\sin40^\circ}{1.33}\Rightarrow\theta_2\approx28.9^\circ$.
\end{solutionbox}
\end{cheatproblem}
\begin{cheatproblem}
  Light moves from glass ($n=1.5$) to air ($n=1$). Find the critical angle for total internal reflection.
\begin{solutionbox}
  $\theta_c=\arcsin\dfrac{n_2}{n_1}=\arcsin\dfrac{1}{1.5}\approx41.8^\circ$.
\end{solutionbox}
\end{cheatproblem}
\end{KnowledgePoint}

\begin{KnowledgePoint}{Thin Lenses and Sign Conventions \DOne}
  \KPFormulas
  \begin{formulabox}
  \textbf{Concept explanation:} Thin lens imaging follows the lens equation with sign conventions; magnification uses image and object sizes/orientations.
  
  \textbf{Core formulas:}
  \[
  \left\{\begin{aligned}
    &\frac{1}{f}=\frac{1}{s}+\frac{1}{s'},\quad m=-\frac{s'}{s}=\frac{h'}{h}.
  \end{aligned}\right.
  \]

  \textbf{Variable definitions:} $f$ focal length; $s$ object distance; $s'$ image distance; $m$ magnification; $h',h$ image/object heights.

  \textbf{Prerequisites \& scope:} Use consistent sign convention (e.g., real is positive); paraxial approximation.
  \end{formulabox}

  \KPHeuristics
  \begin{heuristicsbox}
  \begin{itemize}[leftmargin=*]
    \item Combine equation + ray diagram: draw two principal rays to confirm the algebraic image location.
    \item Remember that negative $m$ indicates inversion; $|m|>1$ indicates magnification.
    \item Sign convention (real-is-positive): take $s>0$ for real objects and $s'>0$ for real images on the opposite side of the lens from the object; $s'<0$ indicates a virtual image on the object side (then $m>0$ and the image is upright).
  \end{itemize}
  \end{heuristicsbox}

  \KPProblems
  % Simple principal-ray diagram (converging lens, object inside f -> virtual image)
  \begin{center}
    \begin{tikzpicture}[x=0.06cm,y=0.06cm,>=Latex]
      % Optical axis
      \draw[-] (-60,0) -- (80,0);
      % Lens at x=0
      \draw[thick] (0,-25) -- (0,25);
      % Focal points at ±60 cm (f=60 cm)
      \fill (-60,0) circle (1.2pt) node[below=2pt] {$F$};
      \fill (60,0) circle (1.2pt) node[below=2pt] {$F'$};
      % Object (arrow) at s=30 cm left of lens (inside focal length)
      \draw[thick,->] (-30,0) -- (-30,18) node[left] {object};
      % Principal ray 1: parallel to axis, refracted as if through F'
      \draw[->] (-30,18) -- (0,18);      % incident
      \draw[->] (0,18) -- (60,0);         % refracted
      % Principal ray 2: through center (undeviated)
      \draw[->] (-30,18) -- (0,0);
      \draw[->] (0,0) -- (30,-18);
      % Back-projections of refracted rays to locate virtual image at s'=-60 cm
      \draw[dashed] (0,18) -- (-60,36);
      \draw[dashed] (0,0) -- (-60,36);
      % Image arrow (virtual, upright, magnified)
      \draw[thick,->,blue] (-60,0) -- (-60,36) node[left] {virtual image};
    \end{tikzpicture}
  \end{center}
\begin{cheatproblem}
  An object at $s=30\,\text{cm}$ forms an image at $s'=-60\,\text{cm}$ using a thin lens. Find the focal length $f$ and magnification $m$.
  \begin{solutionbox}
  Lens equation: $\tfrac{1}{f}=\tfrac{1}{s}+\tfrac{1}{s'}=\tfrac{1}{30}+\tfrac{1}{-60}=\tfrac{1}{60}$, so $f=60\,\text{cm}$.
  Magnification: $m=-s'/s = -(-60)/30=2$.
  Since $f>0$, it is a converging lens. Since $s'<0$, the image is virtual. Since $m>0$, the image is upright.
  \end{solutionbox}
  \end{cheatproblem}
\end{KnowledgePoint}

\begin{KnowledgePoint}{Interference and Diffraction \DOne}
  \KPFormulas
  \begin{formulabox}
  \textbf{Concept explanation:} Coherent sources produce interference patterns; finite apertures make the light diffract, establishing angular scales by wavelength/aperture.

  \textbf{Core formulas:}
  \[
  \left\{\begin{aligned}
    &\text{Double-slit maxima: } d\sin\theta=m\lambda,\\
    &\text{Single-slit minima: } a\sin\theta=m\lambda,\ m=\pm1,\pm2,\dots
  \end{aligned}\right.
  \]

  \textbf{Variable definitions:} $d$ slit separation; $a$ slit width; $\lambda$ wavelength; $\theta$ diffraction angle.

  \textbf{Prerequisites \& scope:} Small-angle approximations $\sin\theta\approx\theta$ valid near axis; coherence required for stable fringes.
  \end{formulabox}

  \KPHeuristics
  \begin{heuristicsbox}
  \begin{itemize}[leftmargin=*]
    \item Map angles to screen positions with $y\approx L\tan\theta\approx L\theta$ for small $\theta$.
    \item To resolve features, compare $\lambda$ to $a$ and $d$ to predict fringe spacing/envelope width.
  \end{itemize}
  \end{heuristicsbox}

  \KPProblems
  \begin{cheatproblem}
  For double-slit with spacing $d$ and wavelength $\lambda$, what is the angle of the $m$-th bright fringe?
  \begin{solutionbox}
  $d\sin\theta=m\lambda \Rightarrow \theta=\arcsin(m\lambda/d)$ (small-angle: $\theta\approx m\lambda/d$).
  \end{solutionbox}
  \end{cheatproblem}
\end{KnowledgePoint}

% (Removed outdated Practice Pointers for Part IV Unit 11)

\Unit[Special relativity ($\gamma$, time dilation, length contraction, $E=mc^2$); photoelectric effect; atomic spectra; nuclear decay/half-life]{Unit 12: Modern Physics}

\begin{KnowledgePoint}{Special Relativity \DOne}
  \KPFormulas
  \begin{formulabox}
  \textbf{Concept explanation:} At high speeds, time dilates and lengths contract; energy–mass equivalence relates rest mass to rest energy.

  \textbf{Core formulas (with proper vs observed):}
  \[
  \left\{\begin{aligned}
  &\gamma=\frac{1}{\sqrt{1-v^2/c^2}},\\
  &\text{Time dilation: } \Delta t=\gamma\,\Delta \tau\ \ (\Delta \tau\text{ proper time in moving clock's frame}),\\
  &\text{Length contraction: } L=\frac{L_0}{\gamma}\ \ (L_0\text{ proper length measured at rest with the rod}),\\
  &\text{Relativistic energy: } E=\gamma mc^2\ (E_0=mc^2),\quad p=\gamma mv,\quad E^2=(pc)^2+(mc^2)^2.
  \end{aligned}\right.
  \]

\textbf{Variable definitions:} $\gamma$ Lorentz factor; $v$ relative speed; $c$ speed of light; $\Delta \tau$ proper time (clock's rest frame); $L_0$ proper length (object's rest frame); $E$ total energy; $E_0$ rest energy; $p$ relativistic momentum.

  \textbf{Prerequisites \& scope:} Inertial frames; $v$ along one axis for simple forms; proper quantities measured in an object's rest frame.
  \end{formulabox}

  \KPHeuristics
\begin{heuristicsbox}
  \begin{itemize}[leftmargin=*]
    \item Label frames (S, S') and identify proper time/length before applying formulas.
    \item [\DTwo] Approximation for $v\ll c$: $\gamma\approx1+\tfrac12(v/c)^2$ (derivation and series methods belong to \DTwo).
  \end{itemize}
  \end{heuristicsbox}

  \KPProblems
  \begin{cheatproblem}
  A spaceship moves at $0.8c$ relative to Earth. What factor relates proper time to dilated time?
  \begin{solutionbox}
  $\gamma=1/\sqrt{1-0.8^2}=\tfrac{5}{3}$.
  \end{solutionbox}
  \end{cheatproblem}
\end{KnowledgePoint}

\begin{KnowledgePoint}{Photoelectric Effect \DOne}
  \KPFormulas
  \begin{formulabox}
  \textbf{Concept explanation:} Electrons emit when photon energy exceeds the work function; the threshold frequency is $f_{th}=\phi/h$.

  \textbf{Core formulas:}
  \[
  \left\{\begin{aligned}
    &K_{\max}=hf-\phi,\quad f_{\text{th}}=\phi/h,\\
  \end{aligned}\right.
  \]

  \textbf{Variable definitions:} $h$ Planck constant; $\phi$ work function.

  \textbf{Prerequisites \& scope:} Idealized models; surface effects and detector thresholds may alter observed $K_{\max}$.
  \end{formulabox}

  \KPHeuristics
  \begin{heuristicsbox}
  \begin{itemize}[leftmargin=*]
    \item Stopping potential depends on frequency (threshold via $f_{\text{th}}=\phi/h$), not intensity.
    \item In a $K_{\max}$--$f$ plot, slope $=h$, vertical intercept $=-\phi$.
    \item Increasing intensity raises saturation current but does not change stopping potential.
  \end{itemize}
  \end{heuristicsbox}

  \KPProblems
  \begin{cheatproblem}
  Light of frequency $f$ hits a metal with work function $\phi$. Write the maximum kinetic energy of ejected electrons.
  \begin{solutionbox}
  $K_{\max}=hf-\phi$.
  \end{solutionbox}
  \end{cheatproblem}
\end{KnowledgePoint}

\begin{KnowledgePoint}{Nuclear Decay Basics \DOne}
  \KPFormulas
  \begin{formulabox}
  \textbf{Core formulas:}
  \[
    N(t)=N_0\,2^{-t/T_{1/2}}=N_0 e^{-\lambda t},\ \lambda=\frac{\ln 2}{T_{1/2}}.
  \]
  \end{formulabox}
  \KPHeuristics
  \begin{heuristicsbox}
  \begin{itemize}[leftmargin=*]
    \item For decay chains, use activity $A=\lambda N$; independent branches superpose exponentials.
    \item Plot $\ln N$ vs $t$ to extract $\lambda$ from the slope.
  \end{itemize}
  \end{heuristicsbox}
\end{KnowledgePoint}

% --- New consolidated Practice Pointers for Part IV ---
\subsection*{Part IV: Optics \texorpdfstring{\&}{&} Modern Physics Practice Pointers}
\addcontentsline{toc}{subsection}{Part IV: Optics \& Modern Physics Practice Pointers}
\begin{itemize}[leftmargin=*]
  \item Physics Bowl Geometric Optics Problem 21 Page: 22
  \item Physics Bowl Geometric Optics Problem 25 Page: 26
  \item Physics Bowl Modern Physics Problem 32 Page: 31
  \item Physics Bowl Geometric Optics Problem 34 Page: 32
\end{itemize}

% (Removed redundant KnowledgePoint: Point Charge Field; content covered in Unit 8 Electrostatics)

\end{document}






